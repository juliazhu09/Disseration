\chapter{Conclusions} \label{ch:conclusion}
\section{Summary of Findings}
In this study, various time series models for forecasting employee-turnover counts were tested, and optimal models for turnover forecasts were identified. The dynamic regression model with the additive trend, seasonality, interventions, and U.S monthly composite leading indicator (CLI) efficiently predicted the turnover with training $R^2=0.77$ and holdout $R^2=0.59$. 

The Cox proportional hazards model along with appropriately chosen internal and external variables led to accurate retirement prediction predictions. In the training sample, training MAPE is approximately 25\% while the predictions in holdout data are approximately 5\%. The key internal variables are division, years of service at hire (YCSH), and age at hire (AGEH). Retirement hazard increases significantly when an individual hit 85 points of retirement credit but then reverts to a lower hazard after age 65. Furthermore, the early-retirement incentive plan the organization implemented in 2008 significantly increases the hazard of retirement, particularly for those with points exceeding 85. In addition, S\&P 500 real earnings are significantly associated with hazard of retirement. Quitting behavior differs significantly by division, occupation, and age at start of service. 2008 early-retirement incentive program is also correlated with a significant amount of quitting. The quitting model performed well with MAPE in the training sample estimated at 15.16\% and at 18.77\% in the holdout sample. In addition, external information, S\&P real corporate dividends, is positively correlated with quitting, indicating that as large private-sector companies' profits increase, the hazard of quitting rises significantly.

This study also identified not only significant factors related to employee turnover but also classification rules based on employee records from four R\&D departments. Four significant factors affecting voluntary turnover are job title, gender, ethnicity, age, and years of service. However, employees' major, education level, and department do not correlate with R\&D employees' quitting behaviors. The decision tree generates nine rules to predict a new employee's tenure. These results can assist managers and human resource departments in developing employee-retention strategies, while also reducing hiring lead time and employee turnover cost.  

\section{Implementation of Employee Turnover Forecasting Models}

The conceptual models for building an employee-turnover forecasting system includes three steps: model installation and forecast-results identification, strategy generation, and model update.  These steps are discussed below.
 
The first step is installing the employee-forecasting model in the HR management system to forecast employee turnover. The forecasting model must be easy to install and have a user-friendly interface. After the model is installed, the quarterly or yearly employee-turnover records are exported from the HR management system. Those records include employees' individual information, such as gender, age, years of service, department, and job title. Then this information is imported into the model, and the predicted employee-turnover number (P) for the next six or twelve months is computed. Based on the production plan and budget, HR calculates the number of employees needed in a future year as the demand (D), identifies the current number of employees (S), and finally, HR can compute the number of employees required in the next year as D-(S-P). The employees predicted to have a high probability of turnover are targeted for HR to determine retention and promotion strategies.  

The second step is modifying employee-retention strategies and building a talent-inventory control system. \citet{moncarz2009} found that effective retention strategies, such as promotions and training, can reduce employee turnover long term and positively influence employee retention and tenure. The significant turnover factors provide managers a clear direction for creating retention strategies in such areas as corporate culture and communication, work environment and job design, promotions, customer contentedness and employee recognition, rewards, and compensation \citep{moncarz2009}. However, some organizations with good retention plans still experience high employee turnover as a result of a high proportion of aging workforce. The optimal employee-inventory management strategy reduces cost and maintains a lean management system's normal operation. Many inventory models, such as the Economic Order Quantity model (EOQ), can be used to manage employee inventory. The EOQ model assumes the demand for the employee inventory occurs at a constant rate; an ordering and setup cost k is incurred when hiring new employees; the cost per-year of holding employee inventory is h; and no shortage of employees is allowed. Realistically, the employee-demand number can be determined by the current workforce number, the total turnover counts, and the future workforce number. The company's production plan and strategy determine the total workforce required, and the employee-turnover forecasting model provides the future turnover number. The EOQ model computes the economic order quantity, which is the economic hiring number in a period. Therefore, the optimal hiring strategy is identified to keep employee inventory stable and to shorten the hiring lead time. 

The final step is updating the employee-turnover forecasting model every five years. Because the model is built on historical data, the internal and external factors, as well as the forecasting results, will change after five years. Therefore, it is essential to import the historical employee turnover data into the forecasting model in order to rerun it so that a professional statistician can identify the significant attributes and optimal model.

In conclusion, the employee-turnover model is necessary for the lean management system, which provides support for the system's normal operation and reduces the cost/waste resulting from employee turnover. Implementing this model requires the cooperation of the HR and department managers as well as the HR and statistical staff. As part of a lean system, this model should be considered as important as Kanban, 5S, and other lean tools.  Not limited to manufacturing organizations, the employee-turnover forecasting model could be relevant to service and government organizations. 

\section{Future Work}
The employee-turnover model is based on employee records exported from HR's management system. However, some key information, like salary, is not accessible. Nevertheless, many researchers have proved that employee salary is a key factor in employee-turnover \citep{griffeth2000}. The forecasting model will continue being developed and improved if additional information, like salary, is available. 

Also, qualitative interview and survey methods help researchers capture employees' opinions regarding such issues as job satisfaction, leadership, organization commitment, and turnover intention. Those opinions combined with employees' records can build a mixed employee-turnover forecasting model. Regardless of the statistical methods used in building a model, interview and survey methods can be more precisely designed for certain factors among different targeted groups. These factors are managers' greatest concerns. For example, when the turnover rate suddenly increases, interview and survey data can quickly identify the important changing factor, like leadership, organizational structure, or job satisfaction. 

Finally, the optimal employee hiring strategy can be explored in industrial and academic areas. Instead of simply applying an EOQ model to determine an optimal hiring number, a more complex optimization model can be developed by considering how to redistribute and retrain current employees to the opening position. This may provide additional value to both academic and industry practitioners.


