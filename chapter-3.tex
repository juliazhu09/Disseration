\chapter{Employee Turnover Forecasting and Analysis Using Survival Analysis} \label{ch:surv}
\section{Introduction}
Employee turnover is a topic that has drawn the attention of management researchers and practitioners for decades because it is both costly and disruptive to the functioning of most organizations \citep{staw1980, mueller1989, kacmar2006}. Human resource analytics systems' key goals include identifying not only factors leading to employee satisfaction and productivity but also turnover's causes and timing \citep{IBM}. For many mature firms with large workforces, an important piece of this puzzle is developing predictive models for both retirement and quitting.  While commercial tools may exist in this space, very little discussion of applied predictive models has appeared in academic literature.  From an operational perspective, the ability to accurately predict turnover across a range of organizations and job types is a highly beneficial to front-line management of these organizations and to their financial, human resources, and actuarial concerns of the company and its supporting partners. The ability to forecast turnover becomes even more valuable in specialized industries and government agencies with long hiring lead times. From a research perspective, a predictive retirement model is a platform that can allow investigators to evaluate both external economic and demographic factors as well as internal policies influencing retirement decisions.
%(any reference for this? Tim Munyon survival thing?)

This study focuses on the behavior of individuals between 2000 and 2012 employed by a large industrial organization located at a single site that provided employees with a defined benefit retirement plan. The study has four objectives: 1) Develop a probabilistic model of the employee lifetime as a function of basic demographic, employment, and external factors.  2) Evaluate the model's aggregate predictive accuracy for one-and two-year time frames as a tool to facilitate planning. 3) Determine the internal and external economic variables' impact on retirement. 4) Quantify an early retirement incentive's impact on retirement behavior. Because of the sampling approach used to collect the data, an integral part of the study was ensuring that the modelling strategy was robust to biases introduced by left truncation and right censoring.

To analyze the data, the Cox proportional hazards models (PH) was used.  The strength of the Cox model is the semiparametric form incorporating a non-parameteric baseline estimate and a parametric term that determines the other factors' relative impact. As discussed in Section \ref{Cox.mod}, this model is not sensitive to truncation bias, can be viewed as a modeling approach for non-homogeneous Poisson processes and can include the covariates' effects that change over time.

\subsection{Motivation for the Study}

The current work bridges the gap between factors involved in individual retirement decision-making and workforce management's importance from the perspective of human resources. Here, the study's focus is on predicting turnover for strategic planning in large organizations, such as government agencies, large corporations, and large academic institutions in order to determine changes in workforce size and to plan for the eventual loss of critical skills. This effort is particularly applicable to corporations that have an older workforce with many employees close to retirement and most  with defined benefit retirement plans. Human resources departments, benefits managers, and actuaries can also use these results to better manage investment resources and plan for the near term.

In many industries with older worker populations, retirement is a major source of HR disruption, causing delays and other problems in processing the work flow. Replacing retired workers can also be a major expense both in HR staff time and recruiting costs. Effective predictive models can help managers identify potential individuals from which department in an organization likely to leave, thus giving the organization more lead time in planning for and recruiting replacements. For example, in large organizations that use skilled workers  in both  white-collar  and  blue-collar  jobs, accurately forecasting openings in the next 6 to 12 months can be extremely beneficial in  maintaining continuity  of operations. Similarly, other firms with younger employee populations may be more affected by employees' quitting and would benefit from predictions. 

While aggregate forecast models of turnover and, more specifically, retirement exist (as discussed in Chapter \ref{ch:timesereis}, such models have limited ability to estimate at division or job-category levels. Such models also do not consider the population's demographics, such as age, years of service, pension type, and potentially numerous other factors that can influence the probability of retirement. By  modeling  the  distribution  of time until retirement at the individual level, much more relevant information can be included, such as age, type of retirement plan,  job classification, organizational division, years of service, pension benefits' details, and individual survey responses if they exist, as well as external social and economic trends that are geographically relevant. This model could not only provide accurate predictions but also give managers and researchers feedback on how different factors and incentives may influence retirement and other HR decisions, such as early retirement incentives.

\section{Literature Review}

Employee turnover is a general term referring to the loss of employees resulting from a wide range of causes, such as retirement, death, quitting, termination, promotion, and reassignment.  Each of these turnover modes has different foundational causes and may be more or less prevalent during different points in one's career.

A considerable amount of work has been done identifying underlying causes of turnover both from the perspective of retirement and of voluntary quitting. \citet{rainlall2004} reviewed theories and summarized crucial factors affecting employee retention and turnover both personally and organizationally.  The factors investigated involved employee needs in terms of individual, family, and cultural values, work environment, responsibilities, supervision, fairness and equity, effort, and employees' development.  Mal-functional leadership is key factor leading to turnover when managers do not provide appropriate training, facilities and leadership. From the retirement's point of view  \citet{Wang2010} summarized key theoretical and empirical research studies on causes of retirement between 1986 and 2010 and identified inconsistent findings drawing on research from a wide variety of social science fields. Most of the work cited in that review focuses on the process of retirement and factors driving the retirement decision from the individual perspective. Most relevant to the current work is research in retirement decision making and human resource management.

%%%%%%%%%%%%%

This study focuses on forecasting, which may be accomplished at the aggregate level or may be broken down by organizational factors or by the mode of loss.  For example, using a similar data source considered here, a time series approach to predict future aggregate turnover based on losses in previous years was discussed in Chapter \ref{ch:timesereis}. Such an approach's weakness is that it does not use the employee population's known characteristics, such as age, skill set, performance evaluations, salary, years of service, and numerous other factors to attempt to predict turnover. Based on the theoretical and empirical findings described in earlier studies  \citep{rainlall2004,Wang2010}, using the factors identified as critical in the retirement and quitting decision-making process, it was expected to improve the ability to predict at both the individual and aggregate levels.

Regression models for lifetime data offer the potential to use internal human-resource data when making predictions. In the organizational and business settings, both academic and professional researchers have used these methods to solve practical problems in the industry. While actuarial scientists have used these methods since their inception to create models in risk and insurance \citep{brockett2008}, researchers in finance have more recently explored using these models to reflect lifetimes of banks \citep{Lane1986}, as well as time until default of financial instruments such as fixed-income securities \citep{leclere2005}.

More relevant to the current work is researchers' application of survival analysis and lifetime data methods to customer relationship management. For example, \citet{lu2002} applied survival analysis techniques to predict customer churn for cellular phone services. Their study provided a tool for telecommunications companies to design retention plans for reducing customer churn. Also, \citet{braun2011} used a hierarchical competing risks analysis to model when and why customers terminate their service by employing the data from a land-based telecommunication services provider.

Until now, however, little has been done in the area of human resources. While \citet{berger1993} considered statistical modelling of tenured faculty's retirement within a university setting, they applying the Bayesian statistical approach to modelling retirement outcomes. More recently, major analytics consulting firms, such as IBM and PWC, have begun offering human resource analytics software and services \citep{IBM,PWC}. Within this area, some consultants have proposed basic survival models for employee churn \citep{briggs2015}; but few details are available, and the complexities of real-world situations tend to be avoided.

Survival analysis has been most frequently used to examine lifetime data-generated engineering (reliability) \citep{lawless2011,meeker2014,holt2011}; medicine and epidemiology \citep{kalbfleisch2011}; and less frequently, social sciences (event studies) \citep{allison2010,long2006}.  The first two fields have motivated most of the theoretical development in these fields. 
In the engineering reliability area, \citet{carrion2010} estimates the time to failure of the pipes in a water supply network's dataset under left-truncation and right-censoring by using the extended Nelson estimator \citep{pan1998}.

These methods have been applied to thousands of epidemiological studies, retrospective biomedical studies, and clinical trials over the past 40 years. For example, \citet{claus1991} investigated the familial risk of breast cancer in a large population-based, case-control study using recurrent lifetime analysis. Those researches found that the risks of breast cancer are a function of women's age.  \citet{Kleinmoeschberger2003} provided a thorough book-length overview of the methods and included specific case studies focusing on medical applications.
\section{Data Preparation}\label{data.desc}

Provided by a large multipurpose research organization in the U.S., the analyzed dataset consisted of 4316 active and 3782 former full-time employees. This population of employees was followed across a 12-year window from November 2000 to December 2012. Records of employees who retired or left before November 2000 or who were hired after December 2012 was truncated from the dataset. In addition, 4316 current employees had no termination date.  The sampling approach taken, capturing only employees active in a fixed window, created two forms of bias-right censored and left truncation-in the sample that must be accounted for. Subjects were right censored if their endpoint (retirement and quitting in this case) was unknown at the time of the study since they were still actively employed. Right-censored observations provide information and should not be dropped but must be analyzed differently than complete observations.  Left truncation results from a failure to include cases that failed before the study window's beginning.  A biased sample resulted because only those cases surviving long enough were represented in a sample. Both of these potential biases are considered in the discussion of models (see Section \ref{bias}).
Several static employee attributes are provided in the list below:
\begin{itemize}
	\item Payroll (PR): hourly, weekly, or monthly payroll
	\item Gender (GENDER): male, female
	\item Division (DIV): used to distinguish the departments, among ten departments. In this study, each division's definition was not static throughout the entire observation period. Over the course of time, divisions could be renamed, reduced, or dismissed in reorganizations. Furthermore, no employees transfer between divisions was recorded. Therefore, for prediction purposes, the division indicates the organization level that an employee was associated with at the time of the final observation.
	\item Occupational Code (OC): a standardized code used to describe the job category within the organization for reporting purposes. These codes include Crafts(C), Engineers (E), General Administrative (G), Laborers (L), General Managers (M), Administrative (P), Operators (O), Scientists (S), and Technicians (T). In this study, occupational codes are highly correlated with payroll category: managers, engineers, administrative, and scientists are on monthly payroll; general administrative employees and technicians are on weekly payroll; and other categories are paid on an hourly basis.
	\item Age at Hire (AGEH): either the age of an employee when most recently hired or that employee's age in November, 2000.
	\item Age at Credit (AGEC): age of an employee at most recent time that employee receives credit for pension
	\item Years of Current Service (YCS): years of service accounting for pension credit
	\item Years of Service at Hire (YCSH): years of service accounting for pension credit at the employee's most recent hiring
	\item Termination Date (TD): the date when an employee left the organization
	\item Termination Type: an employee's reason for leaving the organization, such as retirement (RE) or voluntary quitting (VQ).
	\item Points: the sum of the employee's age and YCS. (YCS can be larger than 0 at hire if an employee has credit from earlier employment in the same organization.)
	%\item Age at end (AGET): the age of an employee at the termination date if employee left the organization before October 31th, 2010, or the age of an employee on October 31st, 2010.
\end{itemize}
These twelve indices were operationalized for testing using a 12-month lag of their one-year averages. This approach ensured that the variable could be useful in forecasting since the one-year lag was known at the time of forecasting. The economic indices were originally reported daily or monthly. The yearly average was computed as the index's average value over the previous 12 months.

\section{Model Development and Evaluation}
This study aimed to develop accurate predictive models of retirement and quitting behavior. The models were used to address several key questions: 

1) How accurately can retirement (quitting) be predicted? 

2) What factors indicate an individual is more likely to retire (quit)? 

3) Which external economic factors are most predictive of retirement (quitting)?

4) What is the magnitude of the impact of an Early Retirement Incentive Programs (ERIP)?

5) How do the tenure and age impact retirement?

6) How many employees, by occupational category and division, will retire (quit) next year?

Survival or lifetime data analysis is used to study the time distribution required for a subject from a population to experience an event such as mechanical failure, death, or recovery. Survival regression models relate lifetime distribution's factors, such as the hazard function, to a linear function of explanatory variables. Statistical survival models are often separated into two categories: parametric survival models and semi-parametric proportional hazards (PH) models or Cox models.  In this study, the Cox PH model was employed to build predictive models of retirement and quitting, to estimate an employee's baseline hazard of retirement or quitting, and to identify significant factors that might impact turnover. The parametric models were inappropriate for this study for several reasons. First, it was unlikely that hazards for events like retirement would match common parametric distributions, such as Weibull or log-normal, since the risk should remain close to zero until the usual range of retirement,  when it spikes and then drops quickly again. Furthermore, as mentioned in the introduction, the sampling scheme used in this data involved several biases, which could most easily be adjusted to use the Cox PH model. The Cox model's third advantage is the ability to incorporate time-dependent covariates.    In the case of the current model, these covariates were required both incorporating the impact of a 2008 early retirement incentive program (ERIP) into the organization and examining the effects of two other variables related to pension benefits. Financial indices that vary with time were also captured using this methodology. A special version of the model known as the competing risks analysis was applied for modeling a population who could experience two types of events (in this case, employee retirement and voluntary quitting). In addition to the model fitting, a simulation study was performed to examine the impact of data bias on the Cox proportional hazard model's forecasting capability.
%There are several reasons for employee leaving the organization: retirement, voluntary quit, layoff, transferring, leave of absence, or death. In this study, retirement and voluntary quit are selected and modelled respectively using competing risks analysis.
\subsection{Missing Data Biases: Right Censoring and Left Truncation}\label{bias}
% what is right censor, how to deal with right censor\\
% what is left truncation how to deal with left truncation\\
Right censoring and left truncation are commonly observed forms of missing data in survival analysis data set. In the current study, the study window was from November 2000 to December 2012 as shown in Figure \ref{fig:1}.
\begin{figure}[htbp]
	\centering
	\includegraphics[width=3.5in]{fig1.png}
	\caption{Right Censoring and Left Truncation}
	\label{fig:1}
\end{figure}
The complete records of active employees during this period were included in the dataset whether or not those employees' tenure began or ended outside the study window. Conversely, those employees whose tenure ended before the study window or whose start date occurred after the study window ended were not included in the study.

Let $T$ be the time at which  an individual experiences the event of interest and let $C$ denote the final time the individual is observed. An observation is called {\it right censored} if $T> C$, indicating that actual event time for the individual is not recorded but is only known to be greater than $C$. Thus, employees active at the end of the observation window are right censored. Right censored observations contained information, although incomplete, about a subject's lifetime and require special treatment in order to draw a proper inference,
\begin{align*}
\delta_i&=
\begin{cases}
1   &\text{if  }  t_i \leq c_i \text{ (uncensored),}\\
0   &\text{if  }  t_i > c_i \text{ (censored),}
\end{cases}
\end{align*}
where $i$ denotes the $i^{th}$ observation, and the event's failure time for $i^{th}$ observation is the minimum time between $t_i$ and $c_i$, i.e., $min(t_i, c_i)$, that is when $c_i <t_i $, $c_i$ is taken as the $i^{th}$ observation's end time in order to do the next analysis.

Left truncation is another interesting artifact of the window sampling scheme. Let $T$ again denote the time that the event of interest occurs, and let $X$ denote the time an individual enters the study. Only the individuals with $T \geq X$ are observed in the study window. Those individuals with $T \leq X$ are referred to as left truncated because they could not be included in this study based on the sampling window as shown in Figure \ref{fig:1}. Left truncation exaggerates the number of longer-life individuals, thus leading to a biased sample as shown in Figure \ref{fig:1}. The longest arrow represents a life span of an employee who was hired in 1950 and retired in 2006. While this employee and any in his cohort who are still active are in the sample, others that began in 1950 but retired in 1998, for example, are not. Hence, in the sampling-window approach the longer someone continues working, the more likely they are to appear in the dataset. Therefore, an overabundance of longer living individuals is seen in the data. Left truncation's presence and the associated bias in the data must be considered to accurately estimate survival \citep{carrion2010}.

\subsection{Cox Proportional Hazards Regression Model} \label{Cox.mod}
%   1. what is cox ph regression model.
%   1. what is cox ph regression model.
The Cox proportional hazards (PH) regression model is the most widely used method for modelling lifetime data. Introduced in Cox's (1972) seminal paper (one of the most cited papers in history) \citep{cox1975}, the Cox PH model is the canonical example of the semi-parametric family of models, specifying a parametric form for the covariates' effect on an unspecified baseline hazard rate, which is estimated non-parametrically. The form of hazard model formula is shown in Equation \ref{eq:cox}:
\begin{equation}
\label{eq:cox}
h(t,x)=h_0(t)e^{(\sum_{i=1}^{k}\beta_ix_i)}
\end{equation}
where $x_i=(x_{i1}, x_{i2}, \ldots, x_{ik})$ are characteristics of individual $i$, $h_0(t)$ is the baseline hazard, and $\beta$  is a vector of regression coefficients.
%    why not parametric. baseline cannot fit to any parametric model\\
%    2. cox regression without/ with time dependent variable\\
%Stuff
The model provides an estimator of the hazard at time t for an individual with a given set of explanatory variables denoted by $x_i$.  In the standard Cox model, the linear combination $\sum_{i=1}^{k}\beta_i x_i$, is not a function of time $t$, and is called time-independent.  If $x_i(t)$ is a function of time the model is called the extended Cox PH model which is discussed in Section \ref{sec:coxt}. A key assumption for the model is the proportional hazards assumption, which assumes that explanatory factors have a strictly multiplicative impact on the hazard function so that different groups maintain a constant hazard ratio at all times. However, the Cox PH regression can be extended to handle non proportional hazards using time-dependent variables or stratification; see \cite{kleinMosch2003}.

The Cox PH regression is "robust" and popular, because the baseline hazard function $h_0 (t)$ is an unspecified function and its estimation can closely approximate the correct parametric model \citep{kleinbaum1998}. Taking both sides of the equation's logarithm, the Cox PH model is rewritten in Equation \ref{eq:coxlog}:
\begin{equation}
\label{eq:coxlog}
\log{h(t,x)}=\alpha(t)+\sum_{i=1}^{k}\beta_ix_i
\end{equation}
where $\alpha(t)=\log{h_0(t)}$. If $\alpha(t)=\alpha$ (i.e. constant), then the model reduces to the exponential distribution. As noted earlier, the general Cox PH model puts no restrictions on $\alpha(t)$. The partial likelihood method is used to estimate the model parameters \citep{allison2010}. %The partial likelihood is estimated b baseline .
%[NEED TO CHECK]

\subsection{Time Dependent Variable and Counting Process}
\label{sec:coxt}
Some explanatory variable values changed over the course of the study. The extended Cox PH regression is a modification of the model, incorporating both unchanging time-independent variables as well as variables that change with time or time-dependent variables,
\begin{equation}
\label{eq:timecovar}
h(t,x)=h_0(t)e^{(\sum_{i=1}^{k_1}\beta_ix_i+\sum_{j=1}^{k_2}\gamma_jx_j(t))}
\end{equation}
where $x=(x_1, x_2, \ldots, x_{k_1}, x_1(t), x_2(t), \ldots, x_{k_2}(t))$, $h_0(t)$ is the baseline hazard occurring when $x=0$, $\beta$ and $\gamma$ are the coefficients of $x$. To fit this model, modifying the partial likelihood is required, and the data set is often presented in a format called the {\it counting process} format in order to facilitate this calculation. The current study considers three internal time dependent variables, which are functions of the individual's characteristics. These variables are described below.
\begin{itemize}
	\item Early Retirement Incentive Program (ERIP): a specific time during the study window, part of the 2008 calendar year, when the organization offered an early retirement incentive program; see \citet{ERIP}. This program was time varying in the sense that it occurred at a different age for each individual in the study population and was set at 0 during the period when no program existed and at 1 during the period when the program did exist.
	\item Points 85 (P85): an indicator that an employee amassed 85 service points, the sum of the employee's years of service and age, thus qualifying that employee for full retirement benefits
	\item Age at 65 (A65): an indicator that an employee qualified for retirement by exceeding the age 65 threshold.
\end{itemize}
P85 and A65 are time-varying variables capturing important changes in an individual's hazard level throughout the study. 
The counting process format allows software packages to handle time-dependent variables by creating multiple intervals for each employee. Each interval is defined so that the time-varying variables are constant within the interval.  For example, for an individual who remained active until the end of the study, achieved 85 points in December 2003, exceeded age 65 in December 2007, and received a retirement incentive in 2008, five records were included for the individual:  November 2000-November 2003; December 2003-November 2007; December 2007; January 2008-December 2008; and January 2009-December 2012, the end of the study.

Financial indices represent another form of time-dependent variables. These indices are referred to as external variables because they depend upon factors external to the employee. In models considering financial indices, monthly observations of these indices are aggregated at the yearly level, leading to a smaller number of time intervals in the counting process format.  Economic variables are included at a one-year lag since retirement and quitting decisions are assumed to occur significantly in advance of the actual event and therefore depend on these indicators' older values. Importantly, using lagged quantities allows the model to be adapted to forecast up to 12 months. In the case of data with external indices, the intervals' start points are defined as max (i.e., hired date, January 1st of a certain year) and the end points are defined as min (i.e., terminated date, December 31st of a certain year).
\subsection{Stratification and Multiple Baselines}
%I. what is stratification model;\\
An alternative for handling non-proportional hazards is stratification. A stratified model allows each data subgroup as defined by a grouping variable to have its own baseline hazard while sharing other variables' parameters. If the proportional hazards assumption holds within these subgroups, then this model produces valid common estimates of variable effects using all the observations. Equation \ref{eq:strata} represents the hazard function for stratum {\it z};
\begin{equation}
\label{eq:strata}
h(t,x,z)=h^z_0(t)e^{(\sum_{i=1}^{k}\beta_ix_i)}
\end{equation}
where $z$ represents the grouping variable, and $h^z\sigma_0(t)$ is a baseline hazard based for stratum z and $\beta_i$ are variables' common effects. Note that the strata variables cannot be the variables in the Cox PH model.
\subsection{Testing the PH Assumption}
Three common approaches are available for testing the proportional hazard assumption's validity. The first approach is investigating the Schoenfeld residuals. A second approach is testing the interaction between time-dependent and time-independent variables in the Cox PH model. The PH assumption is valid if the interaction is not statistically significant ($P>0.05$). Finally, including separate baseline hazards for each stratum that the analyst defines can also capture variation in the hazard rate's changes. See \citet{allison2010,collett2015} for more details on these tests.

%Including a stratified variable, when appropriate, can improve the Cox model's %performance.  The C-statistic is used to compare models with and without %stratification with a higher C value indicating a better model \citep{lemke2012}.

\subsection{Competing Risks}
%  what is competing risks. competing ricks can help forecasting employee retirement and voluntary quit. \\
%  why select these two reasons to model.\\
One of the many nuances observed within this data set is the fact that currently employees can leave employment in several mutually exclusive ways, including quitting voluntarily, being laid off, being dismissed for cause, transferring, retiring, or being unable to continue because of disability or death. A competing risk is an event whose occurrence either precludes the event of interest from occurring or fundamentally alters the probability that the interest will occur \citep{tableman2003}. A competing risks model is a common approach when studying a single mode of leaving, such as retirement, if subjects at risk may also exit through an alternative mode such as quitting. In the current study, when considering retirement as the event of interest and voluntary quitting as a competing risk, all observations were initially included in the study and outcomes being a quitting event were treated as censored, allowing the observed work period to be used informatively.

%%%%%%%%%%%%%%%%%

\subsection{Variable Selection and Model Choice} \label{sec:modelchoice}
In the current study two equivalent time measurements were considered as response variables for modelling:  AGE in years and YCS. Because of including time-varying explanatory variables and the need to estimate the baseline hazard for purposes of forecasting, the data had to be formulated as a counting process, see Section \ref{sec:coxt}. After some consideration, age was deemed the better option for analysis because the more condensed distribution of values allowed more accurate the baseline estimates.


The model selection was initiated by considering DIV, AGEH, GENDER and other time-independent variables as well as ERIP, P85, and A65. Non-significant variables were removed using the criteria that p-values should be less than $.05$ and starting with the largest p-value first, (i.e., backwards selection). This removal continued until only statistically significant variables remained. Stratifying the baseline was tested using the occupational code, which did not improve the model. Finally, external time-varying covariates were tested one at a time to capture economic factors, and the impact on model performance was noted.

\subsection{Baseline Smoothing}
To better predict employee retirement and quitting, baseline smoothing methods were applied to generate a smoothed cumulative hazard function. The Cox PH model generates a non-parametric baseline based on the number of events occurring in the reference group. When no events occur in the certain points $t$, the baseline function for time $t$ is equal to the value at $t^*$, when the event occurs right before time $t$ (given $t>t^*$). Thus, when the data have a high proportion of censors, the model will underestimate the failure probability. The cumulative hazard function was smoothed to see whether a better prediction could be achieved. The smoothing method smooths the cumulative hazard function using SAS proc transreg. The new baseline replaces the original non-parametric baseline and generate the failure probability for each individual. In using a smoothed baseline, the number of predicted events is compared with the original baseline's results. Because the voluntary quitting data has a high proportion (90\%) of censor, the number of predicted events may be more accurate after using a smoothed baseline. On the other hand, since one-third of the terminated employees are retired, the smoothed baseline methods may not make better predictions than the original one; however, it may build a better estimation of the cumulative function at the retirement baseline's tail.       

\subsection{Model Evaluation and Comparison}
%\subsubsection{Measurements}
%AIC, BIC, MAPE, c statistic.\\

To evaluate the models considered in this study, the data were first split into two sets: a training set containing all the observations from 2000-2010, and a testing set containing events that occurred in 2011 and 2012 and that involved the same individuals. The testing (holdout) sample was included to accurately measure how well the model would forecast beyond the observed data. Because the testing data set was not included in the model-fitting process, this out-of-sample evaluation better estimates predictive accuracy.(see \citet{kuhn2013} for further discussion of this approach.)

All the fitted models considered in this study were first evaluated using four statistical criteria: Akaikes Information Criterion (AIC), Schwartzs Bayesian Criterion (SBC), mean absolute percentage error (MAPE), and likelihood-based goodness of fit $G^2$. The optimal  model should  minimize the  values of AIC, SBC, MAPE, and $G^2$ when fit to the training data. In this study, the model performance on the holdout dataset was considered more important than that on the training dataset. AIC and SBC assess model fit by balancing a larger likelihood value with a penalty increasing with the number of variables included. Including the penalty term diminishes the potential for over-fitting \citep{allison2010,hosmer2013}. In this study, these measures were generated automatically using the model-fitting process.

%C-statistics or the area under the receiver operating characteristic (ROC) curve is to test whether the probability of predicting the outcome is better than chance. It ranges from 0.5 to 1.  Models are considered acceptable when the C-statistic is higher than 0.7 \citep{hosmer2013}. C-statistics are calculated by using the predicted failure probability compared with the actual outcomes by SAS proc logistic.
To asses predictive measures such as MAPE and $G^2$ , the event of interest's probability was predicted (e.g., retirement) for each active individual during each calendar year of the training or testing data set. For each employee, the conditional probability of the event occurring between $t_j$ and $t_{j-1}$ was computed, given that the employee was active at time $t_{j-1}$. This probability was calculated using the baseline hazard and coefficients from Cox PH models as shown in Equation \ref{eq:prob} below
\begin{equation}
\label{eq:prob}
\begin{split}% to allgin the equation
P\{t_{j-1}<T<t_j|T \ge t_{j-1}\} &=1-P\{T>t_j|T \ge t_{j-1}\}\\
&=1-\frac{S_k(t_j)}{S_k{(t_{j-1})}}   \\
&=1-\frac{{S_0(t_j)}^{exp(\sum_{i=1}^{p}\beta_ix_i)}}{   {S_0(t_{j-1})}^{exp(\sum_{i=1}^{p}\beta_ix_i)}}
\end{split}
\end{equation}
where $T_k$ is survival time of the $k^{th}$ individual , $t_j$ is a specific time value, $S_k(t) = {S_0(t_j)}^{(\sum_{i=1}^{p}\beta_ix_i)}$ is the survival function for the $k^{th}$ individual, $S_0(t)$ is the baseline function that Cox PH model generated, $x_i$ are the individual explanatory variables, and $\beta_i$ are the respective regression coefficients for the variables.

The yearly predicted number of events is the sum of conditional probabilities given by Equation \ref{eq:prob} as shown in Equation \ref{eq:sum},
\begin{equation}
\label{eq:sum}
\begin{split}
\mathrm{E}(\text{Turnover } t_j)&=\sum{[1-\frac{S(t_{ij}|x_i)}{S(t_{ij-1}|x_i)}]}\\
&=\sum{\big[1-\frac{{S_0(t_{ij})}^{exp(\sum\beta x_i)}}{   {S_0(t_{ij-1})}^{exp(\sum\beta x_i)}}\big]}\\
\end{split}
\end{equation}
where $t_{ij}$ is a specific time value for individual $i$. 
The variability in estimates of yearly turnover is the sum of two quantities shown in Equation \ref{eq:variation},
%%%%%%%%%%%%%%%%%%%%%%%%%%%%%%%%
\begin{equation}
\label{eq:variation}
\begin{split}
\mathrm{Var}(\text{Turnover }  t_j) &=\mathrm{Var}[\sum{Y_i}]\\
&=\mathrm{Var}[\sum{Y_i|P_i}] \\
&=\mathrm{E}[\mathrm{Var}(\sum{Y_i|P_i})]+\mathrm{Var}[\mathrm{E}(\sum{Y_i|P_i})]\\ 
\end{split}
\end{equation}
given, $P_i$ is whether an employee left the organization. Thus, yearly individual turnover variation was modeled as a Bernoulli distribution, $Y_i \sim Bernoulli(P_i)$, thus, $\mathrm{Var}(\sum{Y_i|P_1,...P_n})=\sum{P_i(1-P_i)}$. Therefore, $\mathrm{E}[\mathrm{Var}(\sum{Y_i|P_i})]=\sum{P_i(1-P_i)}$. 
Variation occurs in the estimating individual turnover probabilities's vector $p\approx (\hat{p_1} \dots \hat{p_n})$ derived from the Cox model. Thus, $\mathrm{Var}[\mathrm{E}(\sum{Y_i|P_i})]=\mathrm{Var}[\mathrm{E}(\sum{(1-\frac{{S_0(t_{ij})}^{exp(\sum\beta x_i)}}{   {S_0(t_{ij-1})}^{exp(\sum\beta x_i)}}|P_i}))]$, given, $S(t|x)=exp[-\Lambda(t|x)] $, where $\Lambda(t)$is the cumulative hazard function, and $\hat{S}$ and $\hat{\beta}$ represent random variables that are data functions.
\begin{align*}
\begin{split}
\mathrm{Var}[\mathrm{E}(\sum{Y_i|P_i})]
%&=\big[\mathrm{Var}\sum{\frac{{S_0(t_j)}^{\sum\beta x)}}{{S_0(t_{j-1})}^{\sum\beta x}|P_i}\big]}\\
&=\mathrm{Var}\big[\mathrm{E}\big[\sum{\big[\frac{e^{-\Lambda (t_{ij}|x_i)}}{e^{-\Lambda (t_{ij-1}|x_i)}}|P_i\big]} \big]\big]	\\
&=\mathrm{Var}\big[\mathrm{E}\big[\sum{\big[{e^{(\Lambda (t_{ij-1}|x_i)-\Lambda (t_{ij}|x_i))}|P_i}\big]}\big]\big]	\\
&=\mathrm{Var}\big[\mathrm{E}\big[\sum{\big[{e^{h(t_{ij}|x_i)}|P_i}\big]}\big]\big]		\\
&=\mathrm{Var}\big[\mathrm{E}\big[\sum{\big[{e^{h_0(t_{ij})exp(\sum{\beta x_i})}|P_i}\big]}\big]\big]		\\
%&=\mathrm{Var}\sum{\big[{e^{h_0(t_j)}e^{exp(\sum{\beta x})}|P_i}\big]}		\\
\end{split}
\end{align*}
%For small $x$, $e^{x}\approx 1+x$ leading to,
%\begin{align*}
%\begin{split}
%\mathrm{Var}[\mathrm{E}(\sum{Y_i|P_i})]
%&\approx\mathrm{Var}\sum{\big[\hat{h}_0(t_{ij})exp(\sum{\hat{\beta} x_i})|P_i\big]}		\\
%\end{split}
%\end{align*}
%Recall that
%\begin{eqnarray*}
%\mathrm{Var}(XY)&=&\mathrm{E}(X)^2\mathrm{Var}(Y)+\mathrm{E}(Y)^2\mathrm{Var}(X)
%                + 2\mathrm{E}(X)\mathrm{E}(Y)\mathrm{Cov}(X,Y)\nonumber \\
%                &+& \mathrm{Var}(Y)\mathrm{Var}(Y)+\mathrm{Cov}(X,Y)^2
%\end{eqnarray*}
In this case,  computing $\mathrm{Var}[\mathrm{E}(\sum{Y_i|P_i})]$ was difficult because of the covariates' repetitive nature across observations. Bootstrap and simulation methods were employed to compute model estimates' variance. The bootstrap resampling methods for censored survival data was used to compute the model estimates' standard error using R {\tt boot} package \citep{davison1997}. 
%Davison and Hinkley's method is as follows:
%"For $j=1,..,n,$
%First, $Y_j^{0*}$ is generated from the estimated failure time survival function $S(t)^{exp(\beta x)}$. Second, if censor variable $d_j=0$, set $C^*_j=y_j$, and if $d_j=1$, generate  $C^*_j$ from the conditional censoring distibution given that $C_j\ge y_j$, namely {${\hat{G}(y)-\hat{G}(y_j)}/{1-\hat{G}(y_j)}$}, Then, set $Y^*_j=\mathrm{min}(Y_j^{0*},C^*_j)$, with $D^*_j=1$ if $Y*_j=Y_j^{0*}$ and 0 otherwise." 
Ten thousand datasets were generated using a resampling method involving a high performance computing platform. The data was fitted using the Cox model, and the expected failure number during each year was then computed based on the model. The standard error was computed based on the ten thousand replications; and the 95\% prediction confidence intervals were computed assuming that sum statistic is normally distributed, which follows from the Central Limit Theorem.

MAPE is a common measure for computing predictions' accuracy from a forecast model and is often used to compare models since it measures relative performance \citep{Chu1998}. MAPE is calculated as the average percent deviation of a forecast from the actual observation as shown in Equation \ref{eq:mape}.
The implementation of MAPE for predictions of retirement and quitting used the yearly actual and predicted numbers of events ($y_t$ and $\hat{y_t}$ respectively). The predicted yearly retirement number, $\hat{y_t}$,  was the expected retirement count for a particular year and was computed as the sum of conditional probabilities expressed in Equation \ref{eq:prob} over currently active employees. This number follows from the fact that each employee's probability of retiring could be viewed as an independent Bernoulli random variable.


Although less common in forecasting, $G^2$ is another useful criterion for evaluating a model's prediction of dichotomous events \citep{Simonoff2013}. The calculation takes the form,
\begin{equation}
\label{eq:g2}
G^2=2\sum_{t}[y_t\log{(\frac{\bar{p_t}}{\hat{p_t}}})+ (n_t-y_t)\log{(\frac{1-\bar{p_t}}{1-\hat{p_t}}})]
\end{equation}
where $y_t$ is the number of events in year $t$, $n_t$ is the workforce number in year $t$, $\bar{p_t}={y_t}/{n_t}$ is the observed proportion of events, and $\hat{p_t}={\hat{y_t}}/{n_t} $ is the model's predicted number of events.  Small values of $G^2$ indicate close agreement of observed and predicted numbers of events.

\section{Proportional Hazards Models' Simulation Studies}
The goal of this data analysis was to create a predictive model for retirement and other types of turnover based on an of employees database. Section \ref{bias} pointed out two sources of bias present in the data: left truncation and right censoring.  Because the Cox PH model is estimated using a partial likelihood that depends only on the cases at risk at the specific failure times, estimates of regression coefficients should remain unbiased and efficient in the presence of left truncation and right censoring \citep{Harrell2002}. What is less clear is the impact of the bias on model predictions because  they depend on the baseline, estimated by using non-parametric methods, and the regression coefficients' parametric estimates. In addition, a third ever-present challenge-model selection-can also significantly affect predictions.
 
To better understand the effect of various levels of truncation and censoring on the predictions, three simulation studies were conducted based on Weibull simulated data with varying amounts of bias. The basic setup in all three simulations was the same. Data sets of sizes $n =100, 200, 500,$ $1000, 2000,$ $\text{and } 4000$ were generated from a Weibull regression model, a function of one explanatory variable referring to $age$. In the simulation, $age$ was uniformly distributed from 22 to 70 years, a range mimicking the actual distribution of worker ages observed in the sample. The simulation was used in the context of retirement, and most of the results were also appropriate for quitting.
 
The baseline hazard for Weibull distribution with shape $\alpha$ and scale $\lambda$ is $h_0(t)=\alpha(\lambda)^\alpha t^{\alpha-1}$.  Extending this baseline hazard to a hazard from a proportional hazards regression model for $age$, it is simply multiplied by the exponentiated linear predictor, thus shifting the baseline up or down,
 \begin{equation} \label{eq:weibull}
 \begin{split}% to allgin the equation
 h(t|age) & =h_0(t)exp(\beta \times age) \\
 &=\alpha (\lambda (exp(\beta \times age)^{\frac{1}{\alpha}})^\alpha t^{\alpha-1} \\
 &=\alpha(\tilde{\lambda} )^\alpha t^{\alpha-1}
 \end{split}
 \end{equation}
 where, $\tilde{\lambda}=\lambda (exp(\beta \times age)^{\frac{1}{\alpha}}$.
 
The survival times $T_i$ are randomly generated from the Weibull distribution with shape parameter $\alpha=1.5$ and $\tilde{\lambda}=exp(1.5 +0.025 \times age)^{\frac{1}{\alpha}}$.  It follows that $\lambda = e^1$.
 
 The simulations were performed using the {\tt coxreg} and {\tt phreg} functions from the {\tt R}-package {\tt eha} \citep{eha} for model fitting.  Function {\tt coxreg} performs a standard Cox PH regression using the partial likelihood to fit the model.  The {\tt phreg} function performs a parametric proportional hazards regression using both Weibull, Extreme value (EV) baselines.
\subsection{Simulation 1: Right Censoring} \label{rightcensor.sim1}

The first study focused on understanding right censoring's impact, a significant effect in the turnover data analyzed later because of the many employees that remained active for the entire observation window. For each of the sample sizes above, survival times $T_i$ are simulated from the Weibull distribution as described earlier. Four censoring times $C_j$ are defined as the first, second, third, and fourth (maximum) quartiles of the simulated sample of lifetimes and refer to 75\%, 50\%, 25\% and 0\% censoring proportions, respectively. If the $i^{th}$ observation's survival time $T_i$ is below the censoring time ($C_i$), then the lifetime $T_i$ is observed and the censoring indicator $\delta_i=1$. When the survival time $T_i$ for $i^{th}$ observation is greater than the censoring time ($C_i$), then the censoring time $C_i$ is observed and the censoring indicator $\delta_i = 0$.

The results of 100 simulations at each combination of sample size and censoring proportion are shown on the left side of Table \ref{tab:rightcensor}. Column 1 gives the censoring proportion, column 2 indicates the observed number of events before censoring, and columns 3 and 4 show the average $\beta_{age}$ estimates over 100 simulations for both {\tt coxreg} and {\tt phreg}. The values in columns 5 and 6 are the average estimates of $\lambda$ and $\alpha$ from the parametric fit of {\tt phreg}. The simulation results show that censoring proportion and the number of events are two influential factors in estimating the coefficient. The model overestimates the coefficients of age, $\lambda, \text{ and } \alpha$, when the dataset has a high proportion of censoring. For example, when 75\% of the data are censored with only 25 events, the estimates for three parameters are 0.028, 4.043, and 1.564, respectively, which are the highest among all the estimates. As the event number increases, the estimates approach the actual value. For example, the estimation of age, $\lambda, \text{ and } \alpha$ are close to 0.025, 2.7, and 1.5, respectively, as the number of events exceeds 500.


\begin{table}[htbp]
\scriptsize
	\centering
	\caption{Right Censoring and Left Truncation Simulation Statistics}
	\begin{tabular}{@{}C{1.8cm}lllll||C{1.8cm}lllll@{}}
		\toprule
		\multicolumn{6}{c}{Right Censoring}  & \multicolumn{6}{c}{Left Truncation}                \\ \midrule
		Right Censoring Proportion & Events & $\beta_{coxreg}$ & $\beta_{phreg}$ & $\lambda_{phreg}$
		& $\alpha_{phreg}$ & Left Truncation Proportion & Events &$\beta_{coxreg}$ & $\beta_{phreg}$ & $\lambda_{phreg}$ & $\alpha_{phreg}$ \\
		\midrule
		0\% & 100    & 0.026   & 0.027 & 2.931 & 1.509 & 0\%   & 100  & 0.027 & 0.027 & 2.865 & 1.534 \\
		25\% & 100    & 0.027  & 0.027 & 2.962 & 1.527 & 25\%  & 75   & 0.027 & 0.027 & 2.917 & 1.546 \\
		50\% & 100    & 0.028  & 0.028 & 3.237 & 1.530 & 50\%  & 50   & 0.027 & 0.027 & 2.899 & 1.577 \\
		75\% & 100    & 0.028  & 0.028 & 4.043 & 1.564 & 75\%  & 25   & 0.029 & 0.029 & 3.280 & 1.757 \\
		0\%  & 200    & 0.026  & 0.026 & 2.841 & 1.508 & 0\%   & 200  & 0.025 & 0.025 & 2.777 & 1.506 \\
		25\% & 200    & 0.026  & 0.026 & 2.856 & 1.513 & 25\%  & 150  & 0.025 & 0.025 & 2.756 & 1.515 \\
		50\% & 200    & 0.026  & 0.026 & 2.925 & 1.527 & 50\%  & 100  & 0.025 & 0.025 & 2.825 & 1.532 \\
		75\% & 200    & 0.026  & 0.026 & 3.167 & 1.540 & 75\%  & 50   & 0.026 & 0.026 & 2.927 & 1.572 \\
		0\%  & 500    & 0.025  & 0.025 & 2.731 & 1.500 & 0\%   & 500  & 0.025 & 0.025 & 2.732 & 1.509 \\
		25\% & 500    & 0.025  & 0.025 & 2.718 & 1.508 & 25\%  & 375  & 0.025 & 0.025 & 2.737 & 1.514 \\
		50\% & 500    & 0.025  & 0.025 & 2.744 & 1.514 & 50\%  & 250  & 0.026 & 0.026 & 2.778 & 1.514 \\
		75\% & 500    & 0.025  & 0.025 & 2.787 & 1.525 & 75\%  & 125  & 0.026 & 0.026 & 2.835 & 1.547 \\
		0\%  & 1000   & 0.025  & 0.025 & 2.748 & 1.509 & 0\%   & 1000 & 0.025 & 0.025 & 2.710 & 1.504 \\
		25\% & 1000   & 0.025  & 0.025 & 2.747 & 1.512 & 25\%  & 750  & 0.025 & 0.025 & 2.709 & 1.504 \\
		50\% & 1000   & 0.025  & 0.025 & 2.748 & 1.514 & 50\%  & 500  & 0.025 & 0.025 & 2.715 & 1.506 \\
		75\% & 1000   & 0.026  & 0.026 & 2.844 & 1.509 & 75\%  & 250  & 0.025 & 0.025 & 2.694 & 1.524 \\
		0\%  & 2000   & 0.025  & 0.025 & 2.714 & 1.502 & 0\%   & 2000 & 0.025 & 0.025 & 2.740 & 1.503 \\
		25\% & 2000   & 0.025  & 0.025 & 2.713 & 1.503 & 25\%  & 1500 & 0.025 & 0.025 & 2.731 & 1.502 \\
		50\% & 2000   & 0.025  & 0.025 & 2.742 & 1.500 & 50\%  & 1000 & 0.025 & 0.025 & 2.724 & 1.503 \\
		75\% & 2000   & 0.025  & 0.025 & 2.733 & 1.502 & 75\%  & 500  & 0.025 & 0.025 & 2.718 & 1.508 \\
		0\%  & 4000   & 0.025  & 0.025 & 2.719 & 1.504 & 0\%   & 3999 & 0.025 & 0.025 & 2.720 & 1.500 \\
		25\% & 4000   & 0.025  & 0.025 & 2.718 & 1.505 & 25\%  & 3000 & 0.025 & 0.025 & 2.722 & 1.501 \\
		50\% & 4000   & 0.025  & 0.025 & 2.724 & 1.503 & 50\%  & 2000 & 0.025 & 0.025 & 2.710 & 1.502 \\
		75\% & 4000  & 0.025 & 0.025  & 2.729 & 1.513 & 75\%  & 1000  & 0.025  & 0.025 & 2.703 & 1.503 \\ \bottomrule
	\end{tabular}
	\label{tab:rightcensor}%
\end{table}



\subsection{Simulation 2: Right Censoring with Staggered Entry Times} \label{rightcensor:sim2}

To capture more accurately the data sampling scheme's nuances and complexities and to understand right censoring's impact, the above simulation was modified by staggering the entry times. Starting with the Weibull simulated failure times, offset factor $S$ was added following a uniform distribution from 0 to 10 and representing variation in the employees' starting times within the study window. The event time is equal to the sum of the start point and survival time: $S+T$. The censoring time is a single fixed value that ensures a fixed proportion (25\%, 50\%, and 75\%, respectively) of censored observations.  The observation and censoring indicator are then determined as in the first simulation an individual's survival time being $min(C,T_i+S_i) - S_i$. Because some observations started after the cutoff point (censoring time), the sample sizes varied for different censoring proportions. To ensure a constant sample size, 6000 observations were initially simulated, and 400 whose start points occurred before the censoring point were randomly selected.
\begin{table}[htbp]
	%\renewcommand{\arraystretch}{1.2}
	\scriptsize %table fond
	\centering
	\caption{Right Censoring Simulation's Results Based on Various Start Time}
	\begin{tabular}{ccccccc}
		\toprule
		\multicolumn{1}{c}{\multirow{2}{1.5cm}{Censoring Pct.}}  & \multirow{2}[4]{*}{Events} & \multicolumn{3}{c}{Variable Estimates} & \multicolumn{2}{c}{Predicted Events} \\ \cline{3-7}
		&       & $age$   & $\lambda$ & $\alpha$ &\it{coxreg} & \it{phreg}\\
		\midrule
		0\%   & 400   & 0.025 & 2.694 & 1.508 & 398.52 & 400.44 \\
		25\%  & 400   & 0.026 & 2.802 & 1.518 & 394.24 & 401.72 \\
		50\%  & 400   & 0.026 & 2.828 & 1.514 & 340.73 & 398.51 \\
		75\%  & 400   & 0.025 & 2.821 & 1.518 & 215.92 & 400.80 \\
		\bottomrule
	\end{tabular}%
	\label{tab:right2}%
\end{table}%

The simulation results are shown in Table \ref{tab:right2}. As discussed above, both a correctly parametric proportional hazards regression model with a Weibull baseline {\tt phreg} and a semi-parametric Cox PH regression {\tt coxreg} were fit in order to evaluate differences in efficiency. The estimation of age and $\alpha$ are close to the true values (0.025 and 1.5) when averaged over 100 simulations. Based on 400 events, the estimates of $\lambda$ increase from 2.694 with no censoring to over 2.8 when censoring is above 50\%. In general, the results indicate that right censoring has little impact on coefficient estimation and that the semi-parametric estimates show efficiency similar to that of fully parametric estimates. However, right censoring does affect the baseline function's estimation for the Cox PH model as shown in columns 6 and 7 of Table \ref{tab:right2} and Figure \ref{fig:rightbase}. Panel (a) shows no censoring, and the parametric baseline matches the non-parametric. As time increases and the number at risk initially decreases, the non-parametric estimate deviates from the parametric fit. Panel (b) shows 25\% censoring for 100 replications. As the data range decreases, the non-parametric baseline estimate's duration is more restricted, while the parametric fit can be extrapolated with the usual caveats. In panel (d), the 75\% censoring level, an increased variability is shown with the diminished range because of the more restrictive censoring time. Note that these results are indicative of monotonically increasing hazards and may differ if other baseline distributions are considered.

\begin{figure}[h!]
	\centering
	\subfloat[0\% Right Censoring ]{\label{fig:baseno}\includegraphics[width=0.4\textwidth]{base0.png}}
	\quad
	\subfloat[25\% Right Censoring]{\label{fig:base25}\includegraphics[width=0.4\textwidth]{base25.png}}
	\quad
	\subfloat[50\% Right Censoring ]{\label{fig:base50}\includegraphics[width=0.4\textwidth]{base50.png}}
	\quad
	\subfloat[75\% Right Censoring]{\label{fig:base75}\includegraphics[width=0.4\textwidth]{base75.png}}
	\caption{Baseline Comparison by Various Censoring}
	\label{fig:rightbase}
\end{figure}

Figure \ref{fig:rightdiff}, panels (a-d) illustrate the differences between parametric and non-parametric estimates of the baseline cumulative hazard function, $H_0(t)=-log(\hat{S}(t))$ at different time points over 100 simulated data sets.  Panel (a) indicates that without censoring, overall survival estimates including both baseline and covariates are very similar across parametric and semi-parametric approaches.  As lifetime $t$ increases, deviations between the two estimates increase as a result of both the estimates' cumulative nature and the increasing variability of the non-parametric baseline's hazard estimation, i.e., the Breslow estimator \citep{davison1997,Burr1994}, as the number of observations at risk diminishes over time.

\begin{figure}[h!]
	\centering
	\subfloat[No Right Censoring]{\label{fig:diffno}\includegraphics[width=3in]{diff0.png}}
	\quad
	\subfloat[25\% Right Censoring]{\label{fig:diff25}\includegraphics[width=3in]{diff25.png}}
	\quad
	\subfloat[50\% Right Censored ]{\label{fig:diff50}\includegraphics[width=3in]{diff50.png}}
	\quad
	\subfloat[75\% Right Censoring]{\label{fig:diff75}\includegraphics[width=3in]{diff75.png}}
	\caption{Differences in Estimates of Baseline Cumulative Hazard Functions, Parametric and Non-parametric, for four Levels of Censoring}
	\label{fig:rightdiff}
\end{figure}
When censoring is introduced in panel (b), the range of the x axis is restricted.  Although this restriction varies slightly across simulations, the non-parametric estimator clearly cannot estimate survival probabilities beyond the maximum observed lifetime because of the lack of a parametric model for the baseline. In the current simulation, this estimate was dictated by the censoring time, which was set to ensure a fixed proportion of censored cases. In real studies, such as the one introduced in Section \ref{application}, the population and sampling window dictated the maximum observed lifetime. Panels (c) and (d) reiterate both of these factors. As the censoring level increases to 50\% and finally 75\%, the range of non parametric baseline estimates is further restricted, but the accuracy improves because of the increasing concentration of observations before the censoring time. Because all simulations had 400 observations, 75\% censoring ensured 100 events before the censoring point, which is close to 1, thus providing a very accurate baseline estimate and less dispersion than the uncensored data.  Parametric models do not suffer from limitations on the baseline estimate. It is a visual advantage but requires careful model-selection steps, which introduce other challenges.
% Table generated by Excel2LaTeX from sheet 'Sheet2'
\subsection{Simulation 3: Left Truncation}
Finally, a simulation was preformed to evaluate the impact of left truncation bias on parameter estimation and prediction. Again, a simulated sample of employment times $T_1, \ldots, T_n$ was generated. For each observation, $i$, a uniform random variable $S_i \sim U(0,max(T))$ was then generated, representing the simulated employee starting time. The event time is then $R_i = S_i+T_i$. Figure \ref{fig:trunchist} provides a histogram of the simulated population of $R_i$. Four levels of truncation are introduced by shifting the beginning of the sampling window, $L$ from 0 across the quartiles of $R$ . When the $i^{th}$ observation's starting time, $S_i$, is less than truncation time $l_i$, the observation starting point was reset to $l_i$ which is the first point at which the employee is observed in the sampling window. If $S_i> l_i$, then the employee was first observed at $S_i$, which remained the starting point.  If $R_i>C$, where $C$ represents the end of the observation window, then the employee was still active at the end of the sampling window and that employee' turnover time was censored. To isolate the impact of truncation bias, no censored values were generated in this study.

\begin{figure}[h!]
	\centering
    \includegraphics[width=3.5in]{truncationhist.png}
	\caption{Histogram of Simulated Lifetimes}
 	\label{fig:trunchist}
 \end{figure}

Sample size, censoring proportions, and results follow the same protocol given in Section \ref{rightcensor.sim1}. The simulation's results are shown on the right side of Table \ref{tab:rightcensor}.  As before, the values shown represent average parameter estimates over 100 replications for the Cox PH model and a Weibull PH model. The general pattern suggests that as truncation proportion increases for a fixed number of events, the parameter estimates become slightly biased.  In the Cox PH case when the sample size is 100, the coefficient for age increases from .027 to .029 as the truncation increases from 0\% to 75\%. The scale parameter for the parametric model, {\tt phreg}, also increases with truncation percentage.  For samples of size 100, with $\lambda = e^1 \approx 2.718$ the average estimate increases from $2.865$ with no truncation to $3.280$ with 75\% truncation.  Estimates for $\alpha = 1.5$ increase from 1.534 to 1.757. Such effects disappear as the number of events increase.
\begin{figure}[h!]
	\centering
	\subfloat[No Left Truncation]{\label{fig:ev0}\includegraphics[width=0.4\textwidth]{EV0.png}}
	\quad
	\subfloat[25\% Left Truncation]{\label{fig:ev25}\includegraphics[width=0.4\textwidth]{EV25.png}}
	\quad
	\subfloat[50\% Left Truncation]{\label{fig:ev50}\includegraphics[width=0.4\textwidth]{EV50.png}}
	\quad
	\subfloat[75\% Left Truncation]{\label{fig:ev75}\includegraphics[width=0.4\textwidth]{EV75.png}}
	\caption{Left Truncation Simulation Predictions: Comparison of Actual vs. Cox PH and Parametric PH with EV Baseline}
	\label{fig:leftbase}
\end{figure}

\begin{figure}[h!]
	\centering
	\subfloat[No Left Truncation]{\label{fig:leftno}\includegraphics[height=0.3\textwidth,width=0.4\textwidth]{Predev0.png}}
	\quad
	\subfloat[25\% Left Truncation]{\label{fig:left25}\includegraphics[height=0.3\textwidth,width=0.4\textwidth]{Predev25.png}}
	\quad
	\subfloat[50\% Left Truncation]{\label{fig:left50}\includegraphics[height=0.3\textwidth,width=0.4\textwidth]{Predev50.png}}
	\quad
	\subfloat[75\% Left Truncation]{\label{fig:left75}\includegraphics[height=0.3\textwidth,width=0.4\textwidth]{Predev75.png}}
	\caption{Left Truncation Simulation Results: Comparison of Actual vs. Cox PH and Parametric PH with EV Baseline Predicted Failure Number}
	\label{fig:lefttruncation}
\end{figure}
Figure \ref{fig:leftbase} compares the baseline's non-parametric estimates from the Cox PH model with two parametric proportional hazards fits estimated using $\tt eha$ \citep{eha}.  The first parametric fit is properly specified and assumes a  Weibull distribution as in the prior simulations. The second fit assumes that data follow a type I extreme value distribution.
%[CHECK MANUAL ON THIS].
Unlike the simulation described in Section \ref{rightcensor:sim2}, the non-parametric baseline estimates' duration is not limited by the four left truncation proportions. Both Weibull and Cox PH fits overlap with the Cox baselines' variability showing increased variability with the amount of truncation. EV based fits all significantly underestimate the cumulative hazard indicating a potential risk of parametric modeling. However, it is interesting to note that the EV fits converge slightly with increasing amounts of truncation as a result of extreme-value theory and the theory of exceedances. More information about this phenomenon can be found in \citet{coles2001}.

Figure \ref{fig:lefttruncation} focuses on the number of events that the models predict. Here, 2000 events were generated using this section's protocol. Both parametric and nonparametric models were fit to the data, and this procedure was repeated 100 times. The red vertical line in Figure \ref{fig:lefttruncation} is the average left-truncation time across the simulations for each truncation percentage. Using each model, the expected number of failures was then predicted for each unit's time interval and compared to the observed number of failures. Boxplots indicate the range of observed values across the 100 replications.  The non-parametric Cox PH model tended to slightly overestimate the number of events in each interval, indicating that left truncation did not affect baseline estimation. The convention used to compute the survival probability over each interval explains the overestimation. The baseline survival function is stepwise decreasing, and probabilities are based on the estimate at the most recent previous failure. The baseline slightly overestimates the failure for each individual, and in aggregate produces the observed overestimates. 
%[ADD BY JULIA: The overestimation is because of jagged baseline function which is lack of information of certain points. The survival probability of those points is the previous closest failure point].
Matching parametric estimates are generated using a PH model with extreme value (EV) baseline instead of the true Weibull model. The EV fit's underestimation of the baselines shown in Figure \ref{fig:leftbase} leads to drastic under-predictions of the number of events.


In the situations considered, if sample sizes are large enough, particularly if they exceed 250 events, all parameter estimates show very little bias under both parametric and semi-parametric models. If samples are small, regression coefficients are slightly overestimated in both types of models. For data sets with small numbers of events because of heavy right censoring, parametric models tend to overestimate both shape and scale, leading to an underestimated baseline. Although right censoring's impact on the non-parametric baseline is not directly quantified, Figure \ref{fig:rightdiff} suggests that baseline estimates are often smaller than the parametric versions and become more variable near the maximum lifetime. Simulations show similar impacts of left truncation on covariate and baseline effects.  A major difference is that the non-parametric baseline's length is potentially less limited because of the lack of censoring time. This may not be relevant in real world data, which has both effects.

Although this study has more than 50\% right censoring and an unknown proportion of left truncated cases, it still has more than 1000 retirement events, many with long duration (around 50 years). Based on the simulation, the Cox model is expected to provide very accurate estimates of covariate effects in this case and highly accurate baseline estimates for most employees, particularly those under the age of 70.

\section{Results and Analysis} \label{application}
Constructing predictive models for turnover involves several factors. Among the  variables available  for analysis,  the set  that offers  the maximum predictive power must be chosen (i.e., a model that includes variables providing the best possible predictions on out-of-sample testing data sets and not simply on data used to train the model). The baseline hazard estimate's potential strengths and weaknesses and the impact on prediction accuracy must be evaluated. A bootstrap is a useful tool for assessing uncertainty in model predictions. From an administrative perspective, it is also useful to contemplate the predictors' ERIP implications found to have an impact on prediction.

\subsection{Descriptive Analysis}

As described in Section \ref{data.desc}, the current data provides five demographic and career history variables: PR (hourly, weekly, or monthly payroll); GENDER (M, F); DIV (ten divisions); OC (crafts, engineers, general administrative, laborers, managers, administrative, operators, scientists, technicians); AGEH; and YCSH. Table \ref{tab:descriptive} provides marginal counts of the number of workers in the sample within each category. The occupation codes include four large categories (C, E, M, \&P)  with over 1000 employees observed throughout the data sample; four medium-sized groups with 500-700 employees (G, L, R, \&T); and one small group, S, with 208 employees. Payroll data shows that the largest group is paid monthly, followed by hourly, and weekly. In terms of gender, approximately 72\% of employees are male. Finally, divisions, while not fixed over the employee's life, are distributed in a similar fashion to occupational codes with four larger groups and several of smaller groups.
\begin{table}[htbp]
	\centering
	\scriptsize
	\renewcommand{\arraystretch}{1.2}
	\caption{Descriptive Statistics}
	\begin{tabular}{lrrlrr}
		\toprule
		\textbf{Variable}	& \multicolumn{1}{c}{\textbf{Count}} & \multicolumn{1}{c}{\textbf{N \%}}  &   \textbf{Variable}    & \multicolumn{1}{c}{\textbf{Count}} & \multicolumn{1}{c}{\textbf{N \%}} \\
		\midrule
		\textbf{OC} &       &       & \textbf{GENDER} &       &  \\
		C     & 1295  & 16.0\% & F     & 2296  & 28.4\% \\
		E     & 1361  & 16.8\% & M     & 5802  & 71.6\% \\
		G     & 574   & 7.1\% & \textbf{DIV} &       &  \\
		L     & 613   & 7.6\% & Div1 & 1542  & 19.0\% \\
		M     & 1178  & 14.5\% & Div2 & 751   & 9.3\% \\
		P     & 1621  & 20.0\% & Div3 & 1042  & 12.9\% \\
		R     & 595   & 7.3\% & Div4 & 369   & 4.6\% \\
		S     & 208   & 2.6\% & Div5 & 398   & 4.9\% \\
		T     & 652   & 8.1\% & Div6 & 1199  & 14.8\% \\
		Missing & 1     & 0.0\% & Div7 & 302   & 3.8\% \\
		\textbf{PR} &  &   & Div8 & 823   & 10.2\% \\
		Hourly & 2503  & 30.9\% & Div9 & 404   & 5.0\% \\
		Monthly & 4369  & 54.0\% & Div10 & 1268  & 15.7\% \\
		Weekly & 1226  & 15.1\% &       &       &  \\
		\bottomrule
	\end{tabular}%
	\label{tab:descriptive}%
\end{table}%

\begin{table}[htbp]
	\centering
	\scriptsize
	\caption{Descriptive Statistics 2}
	\begin{tabular}{lrrrrrrr}
		\toprule
		& Count & Mean  & Median & Mode  & Minimum & Maximum & Std. Deviation \\
		\midrule
		\multicolumn{1}{l}{Age at Retire} & 1757  & 59.72 & 60.00 & 62.00 & 49.00 & 84.00 & 4.56 \\
		\multicolumn{1}{l}{Years of Service at Retire} & 1757  & 29.72 & 30.90 & 30.06 & 0.05  & 55.68 & 7.74 \\
		\multicolumn{1}{l}{Points at Retire} & 1757  & 89.44 & 88.67 & 85.47 & 51.05 & 136.66 & 9.15 \\
		\bottomrule
	\end{tabular}%
	\label{tab:descrip2}%
\end{table}%

\begin{figure}[h!]
	\centering
	\subfloat[]{\label{fig:age}\includegraphics[width=0.5\textwidth]{Agetdhist.png}}
	\subfloat[]{\label{fig:point}\includegraphics[width=0.5\textwidth]{pointdhist.png}}
	%\subfloat[Hazard function]{\label{fig:Hazard}\includegraphics[width=0.35\textwidth]{loghazard.png}}
	\caption{Histogram of Age and Point at Retire}
	\label{fig:hist}
\end{figure}

Beyond these demographic and career factors, the models also include behavioral variables derived from policy requirements for retirement and early-retirement incentives occuring throughout the observation period. Histograms of retirement age and accumulated pension points are shown in Figure \ref{fig:hist}. The histogram of retirement age is right skewed and shows an anomalous spike at age 62, which is the mode. Also, 79\% of the employees who retired at 62 also reached or exceeded 85 points.

The average retirement age is 59.72, demonstrating that many individuals retire before 62 and most before age 70. In terms of points (i.e., the sum of years of service plus current age) accrued at time of retirement, an irregular distribution is shown with the vast majority retiring with total points ranging from 85 to 100. Relatively few  employees elect to take a reduced pension and retire with diminished benefits with points below 85. Again, 85 points is the mode, indicating that retiring immediately after becoming fully vested in the pension plan is a popular choice.

\subsection{Retirement Models without External Economic Variables}
Section \ref{sec:modelchoice} discussed two potential response variables suitable for time-until-turnover models: age and years of service from hire. Because of the existence of time-varying variables and the need for a baseline estimate to facilitate predictions, age at retirement is chosen arranging the data in a counting process format (SAS only estimates the baseline if the counting process formulation is used).

%1) Need counting process form for time varying covariates.
%2) Need counting process form to estimate the baseline from CoxPh Model.
%3) Need


%put this graphic in policy discussion part


% Table generated by Excel2LaTeX from sheet 'summary'
%\begin{landscape}
\begin{table}[htbp]
	\centering
	\scriptsize
	\caption{Retirement Models' Statistics}
	\renewcommand{\arraystretch}{1.2}
	\begin{tabular}{cL{3.5cm}lllllrrrr}
		\toprule
		\multicolumn{1}{c}{\textbf{No.}}  &\multicolumn{1}{c}{	\textbf{Model}}&  \multicolumn{1}{c}{\textbf{LR}}    &\multicolumn{1}{c}{\textbf{AIC}}   & \multicolumn{1}{c}{\textbf{SBC}}   & \multicolumn{1}{c}{\textbf{\begin{tabular}[c]{@{}c@{}}Pred. \\ MAPE\end{tabular}}} & \multicolumn{1}{c}{\textbf{\begin{tabular}[c]{@{}c@{}}Holdout \\ MAPE\end{tabular}}} &\multicolumn{1}{c}{\textbf{\begin{tabular}[c]{@{}c@{}}Pred. \\ $G^2$\end{tabular}}} & \multicolumn{1}{c}{\textbf{\begin{tabular}[c]{@{}c@{}}Holdout \\ $G^2$\end{tabular}}}  \\
		\midrule
		1 & DIV GENDER PR OC YCSH AGEH & 1194.30 & 19271.3  & 19377.3 &  39.44 & 56.78 & 381.77 & 85.19 \\
		2 & DIV OC YCSH AGEH & 1193.80  & 19269.8 & 19370.5 &  39.45 & 56.84 & 381.92 & 85.29 \\
		3 & DIV ERIP YCSH AGEH & 1451.60  & 18998  & 19061.6 &  25.91 & 15.51 & 128.75 & 2.64 \\
		4 & DIV ERIP YCSH AGEH OC & 1469.92  & 18995.7  & 19101.7 &   25.91 & 15.24 & 129.47 & 2.56 \\
		% DIV ERIP YCSH AGEH and Strata on OC  & 1426.34 & 14602.1 & 13199.8 & 14602.1 & 13263.4 &  43.45 & 87.92 & 186.89 & 50.00 \\
		%9 models by OC & N/A   & N/A   & N/A   & N/A      &    24.62 & 3.40  & 117.9 & 0.26 \\
		5 & DIV ERIP YCSH AGEH  P85 & 1826.95  & 18624.6  & 18693.6 &  25.59 & 19.04 & 109.17 & 3.40 \\
		6 &DIV ERIP YCSH AGEH  P85 P85*A65 & 1873.69  & 18579.9  & 18654.1 &  25.38 & 7.97  & 111.55 & 0.79 \\
		7 &DIV ERIP YCSH AGEH P85 P85*A65 P85*ERIP & 1881.02  & 18574.6  & 18654.0 & 25.42 & 4.20  & 112.27 & 0.81 \\
		%	Logistic  regression  & 4103.91 & 13618.4 & 9556.53 & 13627.3 & 9752.1 &  28.20 & 31.73 & 232.84 & 117.38 \\
		8 &Time series  & N/A     & N/A     & N/A   &   11.17 & 34.38 & 32.10 & 8.54 \\
		\bottomrule
	\end{tabular}%
	\label{tab:modelstats}%
\end{table}%

Extensive model selection using a variety of metrics including log-likelihood, AIC, BIC, and out-of-sample predictive scoring (MAPE and $G^2$) was then applied to identify key predictive factors in the model as shown in Table \ref{tab:modelstats}. %This is likely due to the fact that age 62 is the earliest age for a person to receive social security retirement benefit.
The first four models listed are all standard Cox PH models with various sets of explanatory variables. Among these four, the third model offers the lowest BIC value (19,061), while the fourth has the lowest AIC (18,996) indicating these two models are the best of this subset. Overall, both models perform almost identically on MAPE and holdout MAPE, which is error when the model is used on the out-of-sample data from 2011 and 2012. $G^2$ and holdout $G^2$ were also nearly identical.  From this perspective, the variable OC seems to provide very little predictive impact.

Models 5-7 were built on model 3 and added various interaction terms. Again, models 6 and 7 are roughly equivalent to the more complex model having a smaller AIC and holdout MAPE, and a very slightly higher predicted $G^2$ and predicted MAPE. In the end, the more complex model 7 was selected and explored because of interest in interpreting the interacting model coefficients.

Stratification is another modeling technique that creates a separate baseline for of a categorical predictor's levels. Creating multiple baselines for subgroups makes the model significantly more complex and can hugely reduce log likelihood and traditional model fit. However, this flexibility can also lead to over-fitting and very poor predictive and holdout fits if the stratification variables are not properly chosen or data is lacking at some levels. This methodology was tested but rejected because of the poor predictive performance in the holdout sample.

Finally, for comparison purposes, a time series prediction model was included as discussed in Chapter \ref{ch:timesereis}. This model was fit to monthly aggregate retirement counts from November 2000 to December 2010 based on the same data used here. The model produced an aggregate forecast by month, leading to a very accurate MAPE on the training data but much poorer out-of-sample performance. It did not provide accurate forecasts by occupational code (OC) because of the small individual samples for each subgroup. Furthermore, including other explanatory factors' effects in this model was impossible.

%why did we choose the ones we did

%The time series model is really good, is it a fair comparison

%Any other discrepancies?

Excluding aggregate economic factors, the optimal modelling variables for predicting age at retirement include DIV, YCSH, and AGEH. In addition, based on the understanding of the retirement program's covenants and parameters, several additional variables increasing the model's predictive power were found. These variables include ERIP, P85, A65*P85 and, and ERIP*P85. A65*P85 moderates the P85 effect's impact after the individual has exceeded 65 years of age. ERIP*P85 moderates the P85 effect's impact while the ERIP is in place.  These variables were introduced in Section \ref{data.desc}.

Table \ref{tab:paraest.} describes the fit parameters and hazard ratios. As noted above, gender, occupational code and payroll category were statistically significant predictors, indicating that employees' gender, job types, and payroll status are not associated with the choice of retirement age conditional on the other variables in the model. 
\begin{table}[htbp]
	\centering
	\scriptsize
	\renewcommand{\arraystretch}{1.2}
	\caption{Parameter Estimates for Retirement Models}
	\begin{threeparttable}
		\begin{tabular}{llL{2.5cm}L{1.5cm}L{2.5cm}L{1.5cm}}
			\toprule
			&       & \multicolumn{2}{c}{\textbf{Model W/O External Variable}} & \multicolumn{2}{c}{\textbf{Model with Real Earnings}} \\
			\hline
			Parameter &   Label & Parameter (Standard Error) & Hazard Ratio & Parameter (Standard Error) & Hazard Ratio \\
			\midrule
			%			division & dir2  & -0.965(0.179)***\tnote{1} & 0.381 &-1.025(0.177)*** & 0.359 \\
			%			division & dir3  & -0.241(0.112)* & 0.786 & -0.246(0.111)* & 0.782 \\
			%			division & dir4  & 0.078(0.195) & 1.081 & -0.039(0.195) & 0.962 \\
			%			division & dir5  & -0.131(0.190) & 0.877 & -0.246(0.190) & 0.782 \\
			%			division & dir6  & 2.136(0.095)*** & 8.463 & 2.252(0.095)*** & 9.511 \\
			%			division & dir7  & 2.435(0.129)*** & 11.418 & 2.437(0.129)*** & 11.437 \\
			%			division & dir8  & 0.864(0.106)*** & 2.373 & 0.816(0.106)*** & 2.261 \\
			%			division & dir9  & -3.023(0.581)*** & 0.049 & -2.774(0.504)*** & 0.062 \\
			%			division & dir10 & 0.793(0.093)*** & 2.211 & 0.709(0.093)*** & 2.031 \\
			%			YCSH  &       & 0.019(0.004)*** & 1.019 & 0.043(0.004)*** & 1.044 \\
			%			ERIP & 1     & 0.859(0.169)*** & .     & 0.942(0.109)*** & . \\
			%			AGEH &       & -0.172(0.013)*** & 0.842 & -0.187(0.013)*** & 0.829 \\
			%			P85   & 1     & 1.435(0.091)*** & .     & 0.682(0.072)*** & . \\
			%			A65*P85 & 1     & -1.610(0.206)*** & .     & -0.662(0.177)*** & . \\
			%			ERIP*P85 & 1     & 0.469(0.179)** & .     & 0.600(0.130)*** & . \\
			DIV & Div2  & -0.965 (0.179)*** & 0.381 & -0.969 (0.177)*** & 0.380 \\
			DIV & Div3  & -0.241 (0.112)* & 0.786 & -0.242 (0.111)* & 0.785 \\
			DIV & Div4  & 0.078 (0.195) & 1.081 & 0.013 (0.195) & 1.013 \\
			DIV & Div5  & -0.131 (0.19) & 0.877 & -0.216 (0.190) & 0.806 \\
			DIV & Div6  & 2.136 (0.095)*** & 8.463 & 2.261 (0.093)*** & 9.594 \\
			DIV & Div7  & 2.435 (0.129)*** & 11.418 & 2.515 (0.128) & 12.363 \\
			DIV & Div8  & 0.864 (0.106)*** & 2.373 & 0.855 (0.106)*** & 2.352 \\
			DIV & Div9  & -3.023 (0.581)*** & 0.049 & -2.731 (0.504)*** & 0.065 \\
			DIV & Div10 & 0.793 (0.093)*** & 2.211 & 0.726 (0.093)*** & 2.068 \\
			YCSH  & 1     & 0.019 (0.004)*** & 1.019 & 0.023 (0.004)*** & 1.023 \\
			ERIP  &       & 0.859 (0.169)*** & .     & 0.489 (0.133)*** & . \\
			AGEH  &       & -0.172 (0.013)*** & 0.842 & -0.193 (0.014)*** & 0.825 \\
			P85   & 1     & 1.435 (0.091)*** & .     & 0.756 (0.073)*** & . \\
			P85*A65 & 1     & -1.610 (0.206)*** & .     & -1.019 (0.197)*** & . \\
			ERIP*P85 & 1     & 0.469 (0.179)** & .     & 0.506 (0.141)*** & . \\
			Real Earnings &       &       &       & 0.013 (0.001)*** & 1.013 \\
			\bottomrule
		\end{tabular}%
		\begin{tablenotes}
			\item[1] * denotes $P<0.05$, ** denotes $P<0.01$, and *** denotes $P<0.001$.
		\end{tablenotes}
	\end{threeparttable}
	\label{tab:paraest.}%
\end{table}%	
P85 is an indicator that a person is eligible for maximum retirement benefits. Naturally, this eligibility has a strong impact on the probability that a person will retire.  From a quantitative perspective,  with all other factors held constant, for those achieving 85 points before the age of 65, the hazard ratio is $e^{1.44} = 4.22$ meaning that the hazard of retirement becomes 4.22 times more likely. While not surprising, this quantification is important in predicting individual and aggregate retirement and reflects the modal spike observed in the histogram in Figure \ref{fig:point}.

An alternative eligibility criterion for retirement occurs when individuals exceed age 65; thus, the hazard increases at this point in an employee's career. Because the response variable in the model is age, this 65-year effect's impact is included in the baseline hazard, which should increase after this point. Figure \ref{fig:basepred} shows the baseline survival and cumulative hazard function for a standard case. Independent of the P85 effect, a further steep increases in the cumulative hazard/decrease in survival between age 62 and 65. By including an interaction between the indicators of age greater than 65 and points greater than 85, A65*P85, how the impact of reaching 85 points changes when a person exceeds regular retirement age can be estimated. In this case, the interaction term is estimated at -1.61, indicating a diminishing effect on the P85 criteria to $e^{1.44-1.61} =0.84$. In addition, as can be seen from the cumulative hazard's trend, the baseline hazard seems to return to a lower level.  Hence, those employees who that exceed both criteria actually have a reduced hazard of retiring over those who have met only the age 65 criterion. If the individual remains on the job after meeting both criteria seemingly indicates a diminished intention to retire.

According to the model, an employee's age at time of hire and their years of service at the time of hiring can also influence retirement. The coefficient estimate for age is -0.17. The reference age is 45.49 (i.e.,the hazard ratio for retirement of an employee who started working at age 46.49 is $e^{-.017}=.84$) indicating a 16\% drop in hazard for each additional year later that an employee started. The employee's survival probability at any time ($t$) can be computed as $S(t)^{1.19} = (S(t)^{e^{0.17}})$ when age at hired is one year below 45.49,  where $S(t)$ is the baseline survival probability for a reference employee of average age at the time of hiring. Moving in the other direction, an employee's survival probability is $S(t)^{0.84}=(S(t)^{e^{-0.17}})$ for a one year increase beyond 45.49 in the employee's starting age. Together, this implies that at any given retirement age, the employee who starts earlier is more likely to retire because of having more years of service and being closer to vesting full benefits (85 points) than an equivalent employee who started working at an older age. Similarly, the employee's years of service at hire show a positive estimate (0.019) with a hazard ratio (1.019) indicating that each year of service at hire beyond the population average of 2.75 is associated with an approximately 2\% increase in the hazard of retirement.  This leads to a survival probability $S(t)^{0.98}$ for a one year decrease in the reference years of service at time of hiring. On the other hand, the survival probability is $S(t)^{1.019}$ for an employee with one year of service above average at the time of hire. Together, age at hire and YCSH effects reflect the intuitive fact that, all else being equal, an employee who has more years of service and therefore is closer to full vesting is more likely to retire.  What is non-intuitive about this finding is that while one might suspect that the effects should be of similar magnitude, the effect of one year difference in age actually seems to have about eight times the impact that one year of previous service does on the hazard of retirement.

\begin{figure}[h!]
	\centering
	\subfloat[Retirement Rate]{\label{fig:ratio}\includegraphics[width=0.4\textwidth]{reratio.png}}
	\subfloat[Prediction w/o Variable ERIP]{\label{fig:nopolicy}\includegraphics[width=0.4\textwidth]{nopolicy.png}}
	\caption{Retirement Rate and Actual vs. Forecast Plot by Preliminary Model without ERIP}
	\label{fig:rerate}
\end{figure}

In the fiscal year 2008, the employer in this study created a temporary early-retirement incentive program (ERIP). The response window for this option was three months; however, other details are unknown. To deal with the increased retirement level during this period, a time-dependent indicator variable was included for each employee who indicated their age when this program was in effect. This indicator's coefficient was 0.86, leading to a hazard ratio of $e^{.86} = 2.21$, which indicates that, on average, an individual's hazard of retirement increased almost 2.2 times during this period. If more information were known about this ERIP's requirements or targets, a more case-specific estimate might be possible. The ERIP's effect is considerable as indicated by the huge uptick in events in 2008 (see Figure \ref{fig:rerate}). Not including this one-time effect in the model might bias the other estimates' parameters considerably.

As an additional step, the ERIP effect was tested on the employees eligible for a pension (i.e., employees with points greater than 85). After adding an interaction term between ERIP and the indicator that a person exceeded 85 points, the hazard ratio for the ERIP increases substantially from 2.21 to 15.85, more than seven times the basic ERIP effect. In contrast, employees who were eligible for only partial retirement are not affected by ERIP because an interaction between ERIP and indicator variables that employees achieve only 65 or 75 points, were not statistically significant.

The DIV variable was also a significant predictor. For analysis, the baseline level was chosen arbitrarily as division 1 so that the baseline determined its hazard rate. Relative to this baseline, divisions 6 and 7 have very high hazard ratios (8.363 and 11.405, respectively), indicating-with other factors being equal-that the employees in division 6 and 7 are much more likely to retire at any age than those in division 1. Conversely, division 9 has a hazard ratio of $e^{-3.023} = .049$, indicating that individuals within this group have 1/20 the hazard of group 1.  This finding may indicate that the division is new and contains younger employees.  In general, differences in retirement rates could be a result of differences in age demographics, departmental and job functions, or departmental leadership.
\begin{figure}[h!]
	\centering
	\subfloat[Survival Function]{\label{fig:Surv}\includegraphics[height=0.3\textwidth,width=0.4\textwidth]{Surv.png}}
	\subfloat[Cumulative Hazard]{\label{fig:Cumh}\includegraphics[height=0.3\textwidth,width=0.4\textwidth]{logcum.png}}
	%\subfloat[Hazard function]{\label{fig:Hazard}\includegraphics[width=0.35\textwidth]{loghazard.png}}
	\caption{Baselines with 95\% Confidence Intervals}
	\label{fig:basepred}
\end{figure}

The baseline survival function and log hazard function are shown in Figure \ref{fig:basepred}. The survival probability is 1 before age 49 as shown in Figure \ref{fig:Surv}, indicating that no employees retire before this age. The survival probability starts to slowly decrease from age 50 to 62. By age 62, the survival probability has decreased by nearly 25\%, indicating that 75\% of employees retire at an age greater than 62 years, assuming that they are at average or baseline levels for other factors included in the model. The survival function's slope decreases sharply at this point, indicating the increased retirement rate for workers between age 62 and 65. After age 65, the probability drops off even more because most of the remaining population retires by age 68 or 69.  Accompanying the survival function is the cumulative hazard ratio's log. Again, the steep rise in the cumulative hazard between age 62 and 65 indicates the increased retirement activity during this period. Afterward, the cumulative hazard levels off, indicating a drop in the instantaneous hazard rate at these future points.
\begin{figure}[h!]
	\centering
	\includegraphics[width=0.5\textwidth]{retire2.png}
	\caption{Actual vs. Forecast Retirement Number for the Model without Financial Index}
	\label{fig:predict}	
	
\end{figure}
\begin{table}[h!]
	\centering
	\scriptsize
	\renewcommand{\arraystretch}{1.2}
	\caption{Predictions Comparisons by Occupational Code (OC)}
	\begin{threeparttable}
		\begin{tabular}{lllllllll}
			\toprule
			& \multicolumn{2}{c}{RET w/o External} & \multicolumn{2}{c}{RET w. External} & \multicolumn{2}{c}{VQ w/o External} & \multicolumn{2}{c}{VQ w. External} \\  \cline{2-9}
			& Training & Holdout & Training & Holdout & Training & Holdout & Training & Holdout \\ \midrule
			Crafts  & 31.3\tnote{1} (34.8)\tnote{2} & 23.0 (32.5) & 27.4 (34.8) & 26.9 (32.5) & 1.5 (1.7) & 1.0 (0.5) & 1.5 (1.7) & 1.0 (0.5) \\
			Engin.    & 17.6 (18.9) & 14.5 (24.0) & 16.9 (18.9) & 18.4 (24.0) & 16.8 (20.4) & 8.2 (7.5) & 16.9 (20.4) & 7.4 (7.5) \\
			Gen. Admin. & 7.9 (8.6) & 12.5 (9.5) & 7.4 (8.6) & 16.4 (9.5) & 3.1 (3.4) & 2.9 (5.5) & 3.0 (3.4) & 2.7 (5.5) \\
			Laborer     & 9.2 (8.1) & 10.5 (9.5) & 8.5 (8.1) & 10.9 (9.5) & 2.1 (2.4) & 1.1 (1.0) & 2.1 (2.4) & 1.0 (1.0) \\
			Manager     & 23.3 (27.0) & 37.5 (36.5) & 20.7 (27.0) & 45.6 (36.5) & 6.6 (7.1) & 3.3 (3.5) & 6.7 (7.1) & 3.1 (3.5) \\
			Prof. Admin.     & 27.9 (28.3) & 44.5 (35.0) & 25.2 (28.3) & 57.0 (35.0) & 11.3 (12.4) & 9.9 (8.5) & 11.4 (12.3) & 9.6 (8.5) \\
			Operator     & 12.2 (11.9) & 10.5 (4.5) & 10.7 (11.9) & 12.3 (4.5) & 1.2 (1.3) & 0.6 (0) & 1.2 (1.3) & 0.6 (0) \\
			Scientist     & 2.8 (3.0) & 2.5 (1.5) & 2.7 (3.0) & 2.9 (1.5) & 1.5 (1.6) & 1.1 (2.0) & 1.5 (1.6) & 1.0 (2.0) \\
			Technician     & 8.8 (8.0) & 8.5 (14.5) & 8.0 (8.0) & 11.5 (14.5) & 2.8 (3.2) & 1.4 (1.5) & 2.8 (3.1) & 1.3 (1.5) \\
			\bottomrule
		\end{tabular}%
	\begin{tablenotes}
		\item[1] the number before the parentheses is average predicted number of events.
		\item[2] the number in parentheses is average of actual events.
		\item[3] training period, Jan. 2001 - Dec. 2010, testing period, Jan. 2011 - Dec. 2013
	\end{tablenotes}
		\end{threeparttable}
		\label{tab:cocscode}
	\end{table}
	
%	% Table generated by Excel2LaTeX from sheet 'Sheet4'
%	\begin{table}[h!]
%		\centering
%		\scriptsize
%		\caption{Predictions by Divisions}
%		\begin{threeparttable}
%			\begin{tabular}{clllllllllll}
%				\toprule
%				Year  & \multicolumn{1}{c}{DIV1} & \multicolumn{1}{c}{DIV2} & \multicolumn{1}{c}{DIV3} & \multicolumn{1}{c}{DIV4} & \multicolumn{1}{c}{DIV5} & \multicolumn{1}{c}{DIV6} & \multicolumn{1}{c}{DIV7} & \multicolumn{1}{c}{DIV8} & \multicolumn{1}{c}{DIV9} & \multicolumn{1}{c}{DIV10} \\
%				\midrule
%				%			2001  & 3\tnote{1} (0)\tnote{2} & 0 (0) & 1 (0) & 0 (0) & 0 (0) & 56 (43) & 11 (2) & 5 (10) & 0 (0) & 5 (7) \\
%				%			2002  & 4 (0) & 0 (0) & 2 (0) & 0 (0) & 0 (0) & 72 (54) & 14 (2) & 6 (38) & 0 (0) & 8 (32) \\
%				%			2003  & 6 (0) & 1 (0) & 2 (0) & 0 (0) & 0 (0) & 89 (44) & 19 (7) & 6 (18) & 0 (0) & 9 (34) \\
%				%			2004  & 9 (0) & 1 (0) & 4 (0) & 1 (0) & 1 (0) & 101 (96) & 26 (29) & 7 (13) & 0 (0) & 10 (12) \\
%				%			2005  & 13 (0) & 1 (0) & 5 (0) & 1 (0) & 1 (0) & 88 (114) & 20 (26) & 8 (18) & 0 (0) & 14 (7) \\
%				%			2006  & 19 (34) & 2 (0) & 8 (0) & 2 (5) & 2 (3) & 58 (105) & 12 (32) & 9 (12) & 0 (0) & 17 (16) \\
%				%			2007  & 23 (59) & 3 (0) & 12 (5) & 3 (7) & 3 (9) & 26 (53) & 3 (10) & 12 (6) & 0 (0) & 24 (20) \\
%				%			2008  & 85 (97) & 14 (23) & 51 (79) & 11 (11) & 12 (13) & 16 (16) & 0 (0) & 45 (24) & 1 (0) & 86 (82) \\
%				%			2009  & 27 (17) & 5 (4) & 15 (21) & 4 (3) & 4 (3) & 0 (0) & 0 (0) & 16 (4) & 0 (0) & 27 (13) \\
%				%			2010  & 32 (25) & 7 (10) & 19 (20) & 6 (4) & 6 (4) & 0 (0) & 0 (0) & 23 (6) & 1 (4) & 32 (21) \\
%				%			2011  & 38 (51) & 9 (15) & 23 (25) & 7 (8) & 7 (9) & 0 (0) & 0 (0) & 28 (12) & 1 (15) & 34 (23) \\
%				%			2012  & 42 (36) & 13 (15) & 30 (15) & 9 (3) & 10 (3) & 0 (0) & 0 (0) & 32 (14) & 1 (8) & 41 (15) \\
%				2001  & 3\tnote{1} (0)\tnote{2} & 0 (0) & 1 (0) & 0 (0) & 0 (0) & 57 (43) & 11 (2) & 5 (10) & 0 (0) & 5 (7) \\
%				2002  & 4 (0) & 0 (0) & 2 (0) & 0 (0) & 0 (0) & 72 (54) & 14 (2) & 6 (38) & 0 (0) & 9 (32) \\
%				2003  & 6 (0) & 1 (0) & 2 (0) & 1 (0) & 0 (0) & 89 (44) & 19 (7) & 6 (18) & 0 (0) & 9 (34) \\
%				2004  & 9 (0) & 1 (0) & 4 (0) & 1 (0) & 1 (0) & 101 (96) & 26 (29) & 7 (13) & 0 (0) & 10 (12) \\
%				2005  & 13 (0) & 1 (0) & 5 (0) & 1 (0) & 1 (0) & 87 (114) & 20 (26) & 8 (18) & 0 (0) & 14 (7) \\
%				2006  & 18 (34) & 2 (0) & 8 (0) & 2 (5) & 2 (3) & 58 (105) & 12 (32) & 9 (12) & 0 (0) & 17 (16) \\
%				2007  & 23 (59) & 3 (0) & 12 (5) & 3 (7) & 3 (9) & 26 (53) & 3 (10) & 12 (6) & 0 (0) & 24 (20) \\
%				2008  & 87 (97) & 14 (23) & 52 (79) & 11 (11) & 12 (13) & 16 (16) & 0 (0) & 45 (24) & 1 (0) & 85 (82) \\
%				2009  & 26 (17) & 5 (4) & 15 (21) & 4 (3) & 5 (3) & 0 (0) & 0 (0) & 16 (4) & 0 (0) & 27 (13) \\
%				2010  & 32 (25) & 7 (10) & 18 (20) & 6 (4) & 6 (4) & 0 (0) & 0 (0) & 23 (6) & 1 (4) & 32 (21) \\
%				2011  & 38 (51) & 9 (15) & 23 (25) & 7 (8) & 7 (9) & 0 (0) & 0 (0) & 28 (12) & 1 (15) & 35 (23) \\
%				2012  & 42 (44) & 13 (16) & 30 (33) & 9 (3) & 10 (7) & 0 (0) & 0 (0) & 32 (21) & 1 (15) & 42 (38) \\
%				
%				
%				\bottomrule
%			\end{tabular}%
%			\begin{tablenotes}
%				\item[1] the number before the parentheses is predicted retirement number.
%				\item[2] the number inside the parentheses is actual retirement number.
%			\end{tablenotes}
%		\end{threeparttable}
%		\label{tab:division}%
%	\end{table}%
	
Aggregate predictions for employee retirement are shown in Figure \ref{fig:predict} and Tables \ref{tab:cocscode}. % \ref{tab:division}.
The model predictions capture not only the fluctuations in actual retirement but also the peak year 2008 when the ERIP was introduced. The out-of-sample predictions for holdout years (2011 and 2012) are very close to the actual number, indicating that the model performs well on both the training and holdout samples. Besides predicting overall retirement, the model can also predict by category. Table \ref{tab:cocscode} columns 1 and 2 show the average actual retirement events (in parentheses) and average yearly predictions by occupational code. These predictions are computed by summing individuals' retirement probabilities by job classification then by year and then averaging. Training (2000-2010) and holdout (2011-2012) periods are reported in columns 1 and 2, respectively. Within the training sample, we see close agreement on average between predictions observed across all job classifications. This close agreement reflects the model's effectiveness but also the fact that this is the average over a longer period and estimates the same data used to train the model. Holdout sample predictions are much more variable, indicating that the estimates are a forecast and are in the shorter averaging period. Nevertheless, some large categories show fairly effective predictions in areas such as manager, laborer, and general administration.  Significant under-predictions are observed in crafts, technician, and engineering; but such forecasts still provide a useful baseline for managers. In contrast, significant over-predictions are observed in professional administration and operator categories. Although both deviations may lead to significant over-hiring, only the operator category was significantly over-predicted. Note that the overestimation in the operator category matches the underestimation in the technician category and that the model may not distinguish between these categories effectively.%[NOW USE BOOTSTRAP TO UNDERSTAND UNCERTAINTY.] [Maybe use average squared deviations]
	 
 \subsection{Retirement Models including External Economic Variables} \label{sec:rewExt}
The external economy's impact on retirement decision-making is a topic of considerable interest. To explore this effects, the lagged versions of several economic factors are included using the counting process data formulation with calendar-year based intervals to set up and test the impact on retirement. Parameter estimates for common effects across models with and without economic variables remain constant, as shown in Table \ref{tab:paraest.}.
	 \begin{table}[h!]
	 	\scriptsize
	 	\centering
	 	\renewcommand{\arraystretch}{1.2}
	 	\caption{Economic Index Test Statistics for Retirement}
	 	\begin{threeparttable}
	 		\begin{tabular}{L{3cm}rrrrr}
	 			\toprule
	 			\textbf{Economic Inidcator} &\multicolumn{1}{c}{\textbf{$\chi^2$}} &  \multicolumn{1}{c}{\textbf{P-value}} & \multicolumn{1}{c}{\textbf{\begin{tabular}[c]{@{}c@{}}Hazard \\ ratio\end{tabular}}}   &  \multicolumn{1}{c}{\textbf{MAPE}} &\multicolumn{1}{c}{\textbf{$G^2$}} \\
	 			\midrule
	 			Without Econmic indicator\tnote{1} &       &       &       & 21.31 & 147.43 \\
	 			MHP & 129.614 & $<.001$ & 1.020  & 17.84 & 220.65 \\
	 			SAMHP & 129.516 & $<.001$ & 1.020  & 17.86 & 221.66 \\
	 			SEMHP & 68.055 & $<.001$ & 1.030  & 23.37 & 178.38 \\
	 			SESAMHP & 67.871 & $<.001$& 1.030  & 23.40 & 179.87 \\
	 			S\&P500 & 13.319 & $<.001$ & 1.001 & 20.79 & 129.83 \\
	 			Dividend & 1.045 & 0.307 & 1.015 & 22.63 & 150.03 \\
	 			Earnings & 84.895 & $<.001$ & 1.016 & 21.40 & 105.06 \\
	 			Consumer Price Index & 5.404 & 0.020 & 1.013 & 21.36 & 133.57 \\
	 			Real Price & 5.522 & 0.019 & 1.000     & 21.22 & 138.72 \\
	 			Real Dividend & 1.925 & 0.165 & 1.022 & 23.39 & 154.84 \\
	 			Real Earnings & 80.358 & $<.001$ & 1.013 & 20.33 & 95.02 \\
	 			Long Interest Rate & 1.539 & 0.215 & 1.082 & 22.03 & 149.01 \\
	 			Unemployment Rate & 32.212 & $<.001$ & 0.849 & 25.08 & 179.49 \\
	 			P/E 10 & 0.041 & 0.839 & 0.998 & 21.31 & 147.97 \\
	 			Wilshire5000 & 22.392 & $<.001$ & 1.028 & 20.83 & 121.58 \\
	 			\bottomrule
	 		\end{tabular}%
	 		\begin{tablenotes}
	 			\item[1] it is the selected model without economic indicator.
	 		\end{tablenotes}
	 	\end{threeparttable}
	 	\label{tab:EI}%
	 \end{table}%
	 \begin{figure}[h!]
	 	\centering
	 	\subfloat[Monthly Housing Price]{\label{fig:MHP}\includegraphics[width=0.5\textwidth]{MHP.png}}
	 	\subfloat[Real Earnings]{\label{fig:Real}\includegraphics[width=0.5\textwidth]{realearnings.png}}
	 	%\subfloat[Hazard function]{\label{fig:Hazard}\includegraphics[width=0.35\textwidth]{loghazard.png}}
	 	\caption{Financial Indices and Retirement Predicting Plot}
	 	\label{fig:EIndex}
	 \end{figure}
	 
 As shown in Table \ref{tab:EI}, among financial indices tested, S\&P500, REAL EARNINGS, and WILSHIRE5000 were statistically significant and also improved the model forecast leading to lower MAPE and $G^2$.  Although the REAL PRICES index was also statistically significant, its coefficient estimate of 0.0004 led to a hazard ratio of 1 indicating little if any practical impact. As shown in Table \ref{tab:EI}, the REAL EARNINGS index is the most important factor among all the indicators, leading to the lowest $G^2$ value and showing the strongest impact on retirement behavior. Figure \ref{fig:Real} plots retirement fluctuation against a one-year lag of Real Earnings. Two highly correlated equity financial indices, S\&P500 and WILSHIRE5000, were also statistically significant with hazard ratios $>1$ indicating that this organization's employee are more likely to retire when the stock market is strong.
	 
Unadjusted Monthly Housing Price (MHP) is another influential index, and its inclusion in the model resulted in the lowest MAPE value. Fluctuations in the lag also correlate strongly with the retirement number as shown in Figure \ref{fig:MHP}. However, the retirement number does not decrease coinciding with the decreased MHP after the 2008 financial crisis. Returning to Table \ref{tab:cocscode} columns 4 and 5, most of the holdout predictions were inferior to the model without external variables, probably because of the increased deviation between MHP and retirement in the post-2009 period.

Table \ref{tab:bootre} shows yearly predictions for retirement models along with bootstrapped confidence intervals. The third column shows the model predictions based on the fit using the Cox PH model. The bootstrap bias estimate indicates the difference between the predictions and the actual mean values. The bootstrap standard error was estimated using 10,000 bootstrap replications. As shown in Table \ref{tab:bootre}, the standard errors range from 3.6 to 15.7 for the retirement model. Although some actual retiring numbers were not included in the 95\% confidence intervals, they are close to the predicted confidence intervals. Also, the bootstrap bias estimates are less than 1, indicating the predicted results generated by Cox PH model are nearly unbiased. The retirement models' prediction confidence intervals are shown in Figure \ref{fig:reboot}.

\begin{table}[]
	\scriptsize
	\centering
	%\renewcommand{\arraystretch}{1.2}
	\caption{Retirement Models' Bootstrapping Results}
	\label{tab:bootre}
	\begin{tabular}{lrrrrrrrrr}
		\toprule
		&        & \multicolumn{4}{c}{RE w/o Real Earnings}                                 & \multicolumn{4}{c}{Re w/ Real Earnings}                                   \\ 
		Year & Actual & Pred.& Bias    & Std Error & $Var\sum{\hat{Y_i}}$        & Pred. & Bias    & Std Error & $Var\sum{\hat{Y_i}}$          \\ \midrule
		2001 & 62  & 80  & -0.0524 & 4.40  & 91.03  & 85  & -0.2888 & 6.12  & 113.40  \\
		2002 & 126 & 105 & 0.1736  & 4.62 & 112.38 & 82  & 0.3079  & 4.79  & 96.58  \\
		2003 & 103 & 134 & 0.0480   & 4.62 & 129.76 & 90  & 0.3889  & 6.10   & 115.29 \\
		2004 & 150 & 160 & 0.2538  & 4.71 & 144.18 & 122 & 0.4728  & 5.22  & 128.09 \\
		2005 & 165 & 151 & 0.4936  & 3.97 & 130.68 & 156 & 0.3703  & 4.99  & 143.42 \\
		2006 & 207 & 129 & 0.3080  & 3.59 & 112.81 & 146 & 0.1932  & 5.55  & 143.24 \\
		2007 & 169 & 109 & 0.2762  & 3.44 & 102.96 & 140 & 0.1213  & 7.18  & 165.28 \\
		2008 & 345 & 324 & 0.9460  & 15.70 & 472.33 & 296 & 0.7722  & 15.54 & 457.43 \\
		2009 & 65  & 99  & 0.1832  & 3.35 & 100.11 & 86  & 0.2380   & 3.54  & 91.60   \\
		2010 & 94  & 124 & 0.2790   & 3.69 & 122.31 & 71  & 0.5767  & 4.88  & 90.09  \\
		2011 & 158 & 148 & 0.3432  & 3.86 & 139.52 & 178 & 0.2685  & 4.48  & 170.08 \\
		2012 & 177 & 180 & 0.4558  & 4.41 & 156.04 & 226 & 0.6810   & 6.95  & 229.06
		\\ \bottomrule
	\end{tabular}
\end{table}	 
	 %\begin{figure}[htbp]
	 % 	\centering
	 % 	\includegraphics[width=1\textwidth]{regain.png}
	 % 	\caption{Lift Chart for Retirement Models}
	 % 	\label{fig:regain}
	 % \end{figure}%
\begin{figure}[]
	\centering
	\subfloat[]{\label{fig:reboot1}\includegraphics[width=0.5\textwidth]{reboot1.png}}
	\subfloat[]{\label{fig:reboot2}\includegraphics[width=0.5\textwidth]{reboot2.png}}
	%\subfloat[Cumulative Hazard]{\label{fig:vqcum}\includegraphics[width=0.35\textwidth]{VQCUM.png}}
	\caption{Retirement Models' Forecast with 95\% Confidence Intervals}
	\label{fig:reboot}
\end{figure}	 
 Another perspective on the survival models' predictive effectiveness is based on looking at lift and gains plots. Figure \ref{fig:regainlift} compares the optimal retirement model  (model 7 from Table \ref{tab:modelstats}) against the same model with the external variable Real Earnings added. The gains plot on the left compares the ranked model fit probabilities (Depth of File) against the cumulative \% of positive response.  Ideally, if the model fits well, a large proportion of retirees will be captured in the first percentiles. For example, as shown in Figure \ref{fig:regainlift}, the 25\% of employees predicted most likely to retire contain close to 75\% of the actual retirees in 2011; furthermore, the top 50\% contain about 95\% of the retirees in observed in that year. This figure also compares models with and without Real Earnings to a random-ordering model. The random-ordering model plotted on the diagonal is a worst-case reference, which essentially produces random predictions. In this baseline case, only 10\% of retirees are gained for each additional 10\% of employees included. Results indicate that the model with Real Earnings included predicts well but is not as effective as the model without those predictions. In either case, the model provides meaningful and accurate predictions of retirement propensity. 
	 \begin{figure}[h!]
	 	\centering
	 	\subfloat[Cumulative Gain Chart]{\label{fig:regain2}\includegraphics[width=0.4\textwidth]{regain3.png}}
	 	\subfloat[Cumulative Lift Chart]{\label{fig:relift2}\includegraphics[width=0.4\textwidth]{relift3.png}}
	 	\caption{Retirement Models: Gain and Lift Charts of Predictive Efficiency for 2011 Holdout Data}
	 	\label{fig:regainlift}
	 \end{figure}
 Figure \ref{fig:relift2} provides another model evaluation perspective, indicating for a fixed sample percentage, how much ”lift” or additional targets, above the baseline percentage is observed in the sample. For example, suppose that 3\% of the employees retired in 2011. If a sample of 10\% of the employees with highest predicted probabilities is considered from model 7, then the lift plot indicates that about four times as many retirees, or about 12\% of this sample, will appear in this group. The model with an external variable performs slightly worse in the first decile, identifying only three times as many retirees or about 9\% of the sample. In either case, the model does a reasonable job predicting retirement probability at the individual level\citep{kuhn2013}.  %[Need to contrast individual predictions and group level.  A reference to the Applied Predictive Modelling, Kuhn \& Johnson 2013]
 
\subsection{Voluntary Quitting Models without External Economic Variables} \label{VQ:woE}

%I. dependent variable are YCSH, because age is not able to predict well. Why plot YCSH by vq and age by vq to describe why.
%one employee quit with 67 years old and less than 85 points. all quit before 65 years.
%II. shorten the length of risk set. how to shorten the risk set.
%iii. model comparison. (survival model, time series model, and logistic regression model)

The second area of interest in analyzing turnover is voluntary quitting. Using the same methodology and set of variables applied to retirement modelling, a predictive model for quitting is constructed.  The challenge of modelling quitting in the current context is that the data has approximately 600 out of 8000 employees quitting during the 10-year study window, resulting in a very high proportion of censored data. With this data density, the model cannot generate a smooth baseline to achieve a good forecasting model using age as the dependent variable because employees quit across such a wide range of ages (20 to 64); see Figure \ref{fig:agevq} and Table \ref{tab:vqmodelstats}. Modelling years of service (YCS) as the dependent variable compresses the reference frame leading to better estimates with employees usually quitting within the first 10 years of service as shown in Figure \ref{fig:ycsvq}. Since employees will not quit if they are eligible for pension, those employees are removed from the risk set when they meet either of the requirements for retirement. With those cases now censored, the indicators P85 and A65 are no longer useful for modeling.
\begin{figure}[h!]
	\centering
	\subfloat[]{\label{fig:agevq}\includegraphics[width=0.5\textwidth]{vq_age_hist.png}}
	\subfloat[]{\label{fig:ycsvq}\includegraphics[width=0.5\textwidth]{vq_ycs_hist.png}}
	%\subfloat[Hazard function]{\label{fig:Hazard}\includegraphics[width=0.35\textwidth]{loghazard.png}}
	\caption{Histogram of Age and YCS at Quitting}
	\label{fig:vqhist}
\end{figure}
Among all the models shown in Table \ref{tab:vqmodelstats}, model 3 with DIV, OC, AGEC, and ERIP as explanatory variables fits best.  A discrete version of the Cox model based on logistic regression also performs well with both models showing similar MAPE and $G_2$ values; see \citet{allison2010} for more details on discrete models.

\begin{table}[htbp]
	\scriptsize
	\caption{Model Assessment for Voluntary Quitting}
	\renewcommand{\arraystretch}{1.2}
	\begin{tabular}{C{0.5cm}L{2cm}L{2.5cm}rrrrrrr}
		\toprule
		\multicolumn{1}{c}{\textbf{No.}} &\multicolumn{1}{L{2cm}}{	\textbf{Dependent Variables}} &\multicolumn{1}{L{2.5cm}}{	\textbf{Independent Variables}}  &\multicolumn{1}{c}{\textbf{LR}}   &\multicolumn{1}{c}{\textbf{AIC}}  & \multicolumn{1}{c}{\textbf{SBC}}   & \multicolumn{1}{c}{\textbf{\begin{tabular}[c]{@{}c@{}}Pred. \\ MAPE\end{tabular}}} & \multicolumn{1}{c}{\textbf{\begin{tabular}[c]{@{}c@{}}Holdout \\ MAPE\end{tabular}}} &\multicolumn{1}{c}{\textbf{\begin{tabular}[c]{@{}c@{}}Pred. \\ $G^2$\end{tabular}}} & \multicolumn{1}{c}{\textbf{\begin{tabular}[c]{@{}c@{}}Holdout \\ $G^2$\end{tabular}}}  \\
		\midrule
		1    &	AGE &	DIV GENDER OC YCSH ERIP &  1226.8  & 5960.1 & 6041.4 & 83.70 & 91.35 & 1213.96 & 197.21 \\
		2    &	YCS & DIV OC AGEC ERIP&  1012.4 & 6848.5  & 6929.7 & 23.90 & 21.05 & 65.63 & 2.48 \\
		%YCS (Dependent) reduced the riskset &     & 1012.4 & 6848.5& 6929.7 & 21.11 & 21.12 & 32.85 & 3.32 \\
		3    & YCS reduced riskset & DIV OC AGEC ERIP &  838.4 & 6911.2 & 6992.5 & 15.16 & 18.77 & 26.25 & 2.08 \\
		4    & Logistic regression  &DIV OC AGEC ERIP&  1870.7& 4712.6  & 4938.6 & 15.98 & 17.55 & 23.03 & 3.21 \\
		5    &	Time series  &  & NA    & NA    & NA  &    26.41 & 61.32 & 30.88 & 18.33 \\
		\bottomrule
	\end{tabular}%
	\label{tab:vqmodelstats}%
\end{table}%

The results suggest that voluntary quitting is influenced by an employee's age at the beginning of their service. The coefficient estimate for age is -0.025. Because the reference age is 35.44, the hazard of quitting for an employee who started working at age 36.44 is $exp(-.025)=.975$ that of an employee starting at age 35.44. For each additional year of age at the beginning of service, the hazard of quitting drops 2.5\% . The employee's survival probability at any time, $t$, can be computed as $S(t)^{1.025} = (S(t)^{e^{0.025}})$ if that employee's age is one year below average, where $S(t)$ is the baseline survival probability for a reference employee of average age at initial employment. Moving in the other direction, the employee's survival probability increases to $S(t)^{0.975}=(S(t)^{e^{-0.025}})$ for a one year increase in the employee's starting age. Together, these findings imply that at any given voluntary quitting age, the employee who starts earlier is more likely to quit than an equivalent employee who starts working later.

The early retirement incentive option (ERIP) also shows a significant impact on an employee's quitting behavior. This indicator's coefficient is 1.111, leading to a hazard ratio of  $e^{1.111} = 3.04$. This ratio indicates that, on average, an individual's hazard of quitting increased by almost three times during this period. It is unclear why an optional early retirement program influenced quitting, but it is possible that the program led to leadership or organizational disruptions or that it simply occurred during 2008, a time of significant economic upheaval because of rapidly changing external economic conditions.
%[NEED CITATION HERE]
\begin{table}[htbp]
	\centering
	\scriptsize
	\renewcommand{\arraystretch}{1.2}
	\caption{Parameter Estimates for Voluntary Quitting Models}
	\begin{threeparttable}
		\begin{tabular}{llL{2.5cm}L{1.5cm}L{2.5cm}L{1.5cm}}
			\toprule
			&       & \multicolumn{2}{c}{\textbf{Model w/o external variable}} & \multicolumn{2}{c}{\textbf{Model with Real Dividend}} \\
			\hline
			Parameter &   Label & Parameter (Standard Error) & Hazard Ratio & Parameter (Standard Error) & Hazard Ratio \\
			\midrule
			DIV & Div1  & -3.066 (0.263)*** & 0.047 & -3.305 (0.269)*** & 0.037 \\
			DIV & Div2  & -2.652 (0.198)*** & 0.071 & -2.875 (0.204)*** & 0.056 \\
			DIV & Div3  & -3.015 (0.253)*** & 0.049 & -3.258 (0.258)*** & 0.038 \\
			DIV & Div4  & -2.533 (0.288)*** & 0.079 & -2.769 (0.292)*** & 0.063 \\
			DIV & Div5  & -2.739 (0.314)*** & 0.065 & -2.934 (0.317)*** & 0.053 \\
			DIV & Div7  & 0.028 (0.145) & 1.029 & 0.101 (0.146) & 1.107 \\
			DIV & Div8  & -0.806 (0.157)*** & 0.447 & -0.968 (0.163)*** & 0.380 \\
			DIV & Div9  & -3.985 (0.586)*** & 0.019 & -4.207 (0.588)*** & 0.015 \\
			DIV & Div10 & -1.325 (0.136)*** & 0.266 & -1.500 (0.142)*** & 0.223 \\
			OC  & C     & -1.163 (0.274)*** & 0.313 & -1.110 (0.275)*** & 0.329 \\
			OC  & G     & -0.794 (0.198)*** & 0.452 & -0.796 (0.198)*** & 0.451 \\
			OC  & L     & -0.711 (0.228)** & 0.491 & -0.619 (0.229)** & 0.539 \\
			OC  & M     & -0.541 (0.150)*** & 0.582 & -0.497 (0.150)*** & 0.609 \\
			OC  & P     & -0.530 (0.135)*** & 0.588 & -0.506 (0.136)*** & 0.603 \\
			OC  & R     & -0.950 (0.308)** & 0.387 & -0.886 (0.309)** & 0.412 \\
			OC  & S     & -0.691 (0.275)* & 0.501 & -0.682 (0.276)* & 0.506 \\
			OC  & T     & -0.608 (0.203)** & 0.544 & -0.564 (0.204)** & 0.569 \\
			AGEC &       & -0.025 (0.005)*** & 0.975 & -0.026 (0.005)*** & 0.974 \\
			ERIP  & 1     & 0.851 (0.135)*** & 2.343 & 0.479 (0.149)** & 1.614 \\
			Real Dividends &       &       &       & 0.085 (0.017)*** & 1.089 \\
			\bottomrule
		\end{tabular}%
		\begin{tablenotes}
			\item[1] * denotes $P<0.05$, ** denotes $P<0.01$, and *** denotes $P<0.001$.
		\end{tablenotes}
		
	\end{threeparttable}
	\label{tab:vqparaest}%
\end{table}

The DIV variable is also a significant predictor. For analysis, the baseline level was chosen arbitrarily as division 6 so that its hazard rate was determined by the baseline.  Relative to this baseline, division 7 has a similar hazard of quitting according to the model estimates.
Conversely, the other divisions have negative coefficients with hazard ratios less than 1, indicating that individuals within these groups have a lower hazard of quitting than group 6. 


The OC variable is another significant predictor. For analysis, the baseline level is engineered for this variable so that its hazard rate is determined by the baseline. Relative to this group, the other divisions have negative coefficients with hazard ratios less than 1. In particular, the crafts group with a coefficient of -1.165 and a hazard ratio of 0.312 shows a much lower hazard of quitting than the engineering group. This finding may reflect differences in sociological or demographic factors among workers in these job categories, differences in compensation relative to other opportunities, or other differences in local or national economic mobility. 


%In general, differences in quitting rates could be caused by differences in age demographics, leadership, departmental and job function, or departmental leadership.
The baseline survival function and log hazard function for quitting are shown in Figure \ref{fig:vqbasepred}. The survival probability decreases steeply between 0 and 10 years of service. At 10 years of service, the survival probability has decreased to close to 0.25, indicating that 75\% of employees quit within the first 10 years of service. From 10 to 30 years of service, the survival function's slope flattens to 0. Accompanying the survival function is the cumulative hazard ratio's log.  Again, the cumulative hazard's steep rise between 0 and 10 year of service indicates the increased quitting activity during this period. After ten years the cumulative hazard levels off, indicating a drop in the hazard rate at these future points.
\begin{figure}[h!]
	\centering
	\subfloat[Survival Function]{\label{fig:vqsurv}\includegraphics[width=0.4\textwidth]{VQSURV.png}}
	\subfloat[Log of Cumulative Hazard]{\label{fig:vqcumh}\includegraphics[width=0.4\textwidth]{VQlogcumhaz.png}}
	%\subfloat[Cumulative Hazard]{\label{fig:vqcum}\includegraphics[width=0.35\textwidth]{VQCUM.png}}
	\caption{Voluntary Quitting Model's Baselines with 95\% Confidence Intervals}
	\label{fig:vqbasepred}
\end{figure}

Predictive results for voluntary quitting without external variables appear in Table \ref{tab:cocscode} columns 6 and 7. The quitting volumes are much lower than retirements in general, but average predictions are highly accurate for both training and holdout with the most sizeable deviation being general administration's under prediction by 2.6 units on average.

\subsection{Voluntary Quitting Models including External Economic Variables}
%i. tested which variable does significantly impact on employee voluntary quit.
Because voluntary quitting is more sensitively affected by macroeconomics, the external variables are further examined using the counting process model with yearly interval based on the calendar year to test their effects on voluntary quitting. This model has the similar parameter estimation as the selected model as shown on the right side of Table \ref{tab:vqparaest}.
\begin{table}[h!]
	\scriptsize
	\centering
    \renewcommand{\arraystretch}{1.2}
	\caption{Economic Index Test Statistics for Voluntary Quitting}
	\begin{threeparttable}
		\begin{tabular}{@{}llllll@{}}
			\toprule
			\textbf{Economic Inidcator}\tnote{1} &\multicolumn{1}{c}{\textbf{$\chi^2$}} &  \multicolumn{1}{c}{\textbf{P-value}} & \multicolumn{1}{c}{\textbf{\begin{tabular}[c]{@{}c@{}}Hazard \\ ratio\end{tabular}}}   &  \multicolumn{1}{c}{\textbf{MAPE}} &\multicolumn{1}{c}{\textbf{$G^2$}} \\ \midrule
			Without Econmic indicator &            &                &              & 15.71 & 28.33 \\
			MHP                   & 37.93      & \textless.001  & 1.012        & 13.62 & 18.44 \\
			SAMHP                    & 37.75      & \textless.001  & 1.012        & 13.64 & 18.50 \\
			SEMHP         & 39.80      & \textless.001  & 1.020        & 13.74 & 18.67 \\
			SESAMHP          & 39.63      & \textless.001  & 1.020        & 13.75 & 18.68 \\
			S\&P500                     & 0.02       & 0.879          & 1.000        & 15.46 & 27.79 \\
			Dividend                  & 31.21      & \textless.001  & 1.077        & 13.01 & 18.01 \\
			Earnings                  & 8.83       & 0.003          & 1.009        & 15.81 & 23.88 \\
			Consumer Price Index      & 35.84      & \textless.001  & 1.024        & 24.26 & 42.95 \\
			Real Price                & 5.34       & 0.021          & 1.000        & 17.94 & 34.52 \\
			Real Dividend             & 26.66      & \textless.001  & 1.089        & 12.40 & 16.37 \\
			Real Earnings             & 3.71       & 0.054          & 1.005        & 14.16 & 23.34 \\
			Long Interest Rate        & 12.04      & 0.001          & 0.789        & 19.10 & 38.73 \\
			Unemployment Rate         & 2.99       & 0.084          & 1.068        & 17.04 & 34.96 \\
			P/E 10                     & 16.22      & \textless.001  & 0.968        & 17.95 & 35.19 \\
			Wilshire5000              & 10.75      & 0.001          & 1.031        & 15.84 & 21.31 \\ \bottomrule
		\end{tabular}
		\begin{tablenotes}
			\item[1] it is the selected model without economic indicator.
		\end{tablenotes}
	\end{threeparttable}
	\label{tab:vqEI}%
\end{table}

\begin{figure}
	\centering
	\includegraphics[width=0.5\textwidth]{realdividend.png}
	\caption{Actual vs. Forecast by Voluntary Quitting Model with Real Dividend and Its Index Plot}
	\label{fig:vqrealdividend}
\end{figure}
One indicator is statistically significant and also improves the model forecasting because of a lower MAPE and $G^2$ than the selected model's values without economic indicator, which is Real Dividend as shown in Table \ref{tab:vqEI}. According to the estimates in Table \ref{tab:EI},Real Dividend is the most important among all the indicators because it has the lowest G2 and MAPE. The test results show that Real Dividend has a strong impact on quitting behaviors. As shown in Figure \ref{fig:vqrealdividend}, the voluntary quitting's plotted fluctuation corresponds with the trend of Real Dividend's one-year lag.

Predictive results for voluntary quitting with external variables appear in Table \ref{tab:cocscode} columns 8 and 9. The predictions closely match those of the voluntary quitting model, which did not include external variables discussed in Section \ref{VQ:woE}, thus indicating that while the model seems highly effective, the external variables' presence does not seem to substantially improve the predictions. 

Table \ref{tab:bootvq} shows the voluntary quitting model's yearly predictions along with bootstrapped confidence intervals. The third column shows the model predictions based on the fit using the Cox PH model.  The bootstrap bias estimate indicates the difference between the predictions and actual mean values. Using 10000 bootstrap replications, the bootstrap standard error is estimated. As shown in Table \ref{tab:bootvq}, the standard errors range from 2.1 to 7.8 for the voluntary quitting model. Although some actual quitting numbers are not included in the 95\% prediction confidence intervals, they are close to the predicted confidence intervals.  Also, the bootstrap bias estimates are all less than 1, indicating the predicted results generated by the Cox PH model are almost unbiased. The confidence intervals for voluntary quitting forecasting are shown in Figure \ref{fig:vqboot}.
\begin{table}[]
	\centering
	\scriptsize
	%\renewcommand{\arraystretch}{1.2}
	\caption{Voluntary Quitting Models' Bootstrapping Results}
	\label{tab:bootvq}
	\begin{tabular}{lrrrrrrrrr}
		\toprule
		&        & \multicolumn{4}{c}{VQ w/o Real Dividend}                                 & \multicolumn{4}{c}{VQ w/ Real Dividend}                                   \\ 
		Year & Actual & Pred.& Bias    & Std Error & $Var\sum{\hat{Y_i}}$        & Pred. & Bias    & Std Error & $Var\sum{\hat{Y_i}}$          \\ \midrule
		2001                 & 52     & 49              & 0.1948  & 2.79                & 56.49               & 46              & -0.0098 & 2.71       & 51.05               \\
		2002                 & 44     & 45              & 0.1256  & 2.30                 & 50.26               & 37              & -0.0364 & 2.47       & 41.35               \\
		2003                 & 48     & 51              & 0.2028  & 2.66                & 57.83               & 41              & -0.0549 & 2.89       & 46.69               \\
		2004                 & 49     & 57              & 0.3166  & 2.80                 & 64.83               & 47              & -0.005  & 3.06       & 52.19               \\
		2005                 & 63     & 53              & 0.1689  & 2.63                & 59.70                & 50              & -0.0889 & 2.58       & 51.03               \\
		2006                 & 65     & 46              & 0.2658  & 2.58                & 52.84               & 50              & -0.0815 & 2.54       & 50.41               \\
		2007                 & 60     & 39              & 0.1600  & 2.28                & 44.48               & 48              & 0.0376  & 2.84       & 51.23               \\
		2008                 & 74     & 69              & -0.3258 & 7.78                & 129.59              & 69              & 0.2345  & 7.35       & 117.52              \\
		2009                 & 40     & 29              & 0.1049  & 2.23                & 34.24               & 47              & 0.0522  & 4.70        & 66.56               \\
		2010                 & 38     & 30              & 0.0827  & 2.14                & 34.35               & 35              & -0.0600   & 2.42       & 40.04               \\
		2011                 & 24     & 29              & 0.1383  & 2.25                & 34.01               & 25              & -0.1132 & 2.09       & 28.96               \\
		2012                 & 36     & 30              & 0.1514  & 2.26                & 35.05               & 30              & -0.0774 & 2.15       & 33.88            
		\\ \bottomrule		     
	\end{tabular}
\end{table}

\begin{figure}[]
	\centering
	\subfloat[]{\label{fig:vqboot1}\includegraphics[width=0.5\textwidth]{vqboot1.png}}
	\subfloat[]{\label{fig:vqboot2}\includegraphics[width=0.5\textwidth]{vqboot2.png}}
	\caption{Voluntary Quitting Models' Forecast with 95\% Confidence Intervals}
	\label{fig:vqboot}
\end{figure}
\begin{figure}[h!]
	\centering
	\subfloat[Cumulative Gain Chart]{\label{fig:vqgain2}\includegraphics[width=0.4\textwidth]{vqgain2.png}}
	\subfloat[Cumulative Lift Chart]{\label{fig:vqlift2}\includegraphics[width=0.4\textwidth]{vqlift3.png}}
	\caption{Voluntary Quitting Models: Gain Chart and Lift Chart for 2011 Holdout}
	\label{fig:vqgainlift}
\end{figure}

As with the retirement models, lift and gains plots can also be used to assess the predictive models effectiveness for quitting by using out-of-sample predictive accuracy. Figure \ref{fig:vqgainlift} compares the optimal retirement model (model 3 from Table \ref{tab:vqmodelstats}) against the same model with the external variable Real Dividends added. Gains and lift charts are interpreted as they were in the retirement case discussed in Section \ref{sec:rewExt}. The gains figure indicates that both models with and without Real Dividends perform with similar predictive accuracy at the individual level, indicating that including the external variable has little practical impact on the model. The number of employees quitting during the sample window is lower than observed for retirement; and, given the information provided by the data, predicting quitting at the individual level is more difficult.

After sorting predicted probabilities, among the 25\% of employees predicted most likely to quit in the 2011 holdout sample, 50\% of the actual quitting cases were observed. Furthermore, the top 50\% of predictions contained about 66.7\% of the quitting cases, and the top 60\% contained close to 75\% of the actual cases. Results indicate that the model including Real Dividends predicts well but is not as effective as the model without those dividends. In either case, the model does provide predictions of quitting propensity that can be useful.

%Figure \ref{fig:vqgain2}, the lift plot, shows that top 25\% of employees with high %predicted retiring probabilities contains around 50\% of quitters, top 50\% contains about %66.7\% quitters, top 60\% contains 75\% of quitters, and top 70\% contains 91\% quitters %in year 2011 for two models, compared to the baseline.

Figure \ref{fig:vqlift2} provides lift estimates, indicating that among the files top 25\%, model 3 captures twice as many of the actual quitting cases as a random sample of the same size.  The model including Real Dividends performs similarly. Based on this finding, the model provides a useful target group for human resource administrators, allowing them to modify and focus their retention strategy for maximum effect. It also identifies an enriched population for management researchers to study the reasons why workers decide to quit.
\subsection{Baseline Smoothing Results}
\subsubsection{Retirement Baseline Smoothing Results}
The selected retirement model without the financial indices baseline was smoothed. As shown in Figure \ref{fig:rebasm}, a smoothed cumulative hazard function was generated across the original cumulative hazard. The smoothed values are close to the actual when age is from 50 to around 70 years old because most employees retired before they were70 years old. After age 70, the smoothed baseline slightly over estimates the actual value, and it greatly underestimates the cumulative hazard when an employee retires at age 84. The predicted results showed the smoothed baseline always predicts a larger number than the original baseline as shown in Figure \ref{fig:repredsm}. The prediction has a smaller gap between the smoothed baseline and the actual retiring number at 2008. However, the prediction has a larger deviation using the smoothed baseline than the one without it in the other years. Furthermore, the training and holdout $G^2$ are 103.1 and 5.7, and overall $G^2$ is 108.8. Also, the training and holdout MAPE are 27.3\% and 12.5\%. Recall $G^2$ in Table \ref{tab:modelstats}, the selected model's $G^2$ values are 111.6 and 0.8 for training and holdout,respectively. The smoothed model performed better in training dataset, but over-predict in the holdout sample. Therefore, the baseline smoothing methods did not provide a better prediction for retirement model.

\begin{figure}[h!]
	\centering
	\subfloat[Smoothed Cumulative Hazard Function]{\label{fig:rebasm}\includegraphics[width=0.35\textwidth]{rebaseline_smooth.png}}
	\subfloat[Retirement Forecast Comparison: Original vs. Smoothed Baselines ]{\label{fig:repredsm}\includegraphics[width=0.35\textwidth]{repred_smooth.png}}
	\caption{Retirement Model's Smoothed Baseline Plot and Forecast Comparison by Original and Smoothed Baselines}
	\label{fig:rebaselinesm}
\end{figure}
\subsubsection{Voluntary Quitting Baseline Smoothing Results}
The cumulative function was smoothed from the selected voluntary quitting model. In Figure \ref{fig:vqbasm}, the solid line is the smoothed cumulative function, and the dot is the original voluntary quitting model. The smoothed values are lower than the original values in the first year. Also. the smoothed function underestimates from 7 to 10 years of service. Other than those years, the smoothed function is close to the original one. Figure \ref{fig:vqpredsm} compares the prediction using the smoothed baseline and the original baseline. The smoothed baselines prediction always surpassed the original baseline. Only in 2008, the smoothed baselines estimation is closer to the actual value.  In the other years, this estimation deviates more from the actual values. Furthermore, the training and holdout $G^2$ are 27.8 and 2.7, respectively, and the overall $G^2$ is 30.4. Also, the training and holdout MAPE are 17.6\% and 22.2\%. Recall $G^2$ in Table \ref{tab:vqmodelstats}, the selected model's $G^2$ values are 26.36 and 2.8 for training and holdout,respectively. The smoothed model did not perform better in either training or holdout dataset. Therefore, the baseline smoothing method attempts to overestimate the actual turnover values in both the retirement and voluntary quitting models in this study.
\begin{figure}[h!]
	\centering
	\subfloat[Smoothed Cumulative Hazard Function]{\label{fig:vqbasm}\includegraphics[width=0.35\textwidth]{vqsmoothbase.png}}
	\subfloat[Voluntary Quitting Forecast Comparison: Original vs. Smoothed Baselines ]{\label{fig:vqpredsm}\includegraphics[width=0.35\textwidth]{vqpred_smooth.png}}
	\caption{Voluntary Quitting Model Smoothed Baseline Plot and Forecast Comparison by Original and Smoothed Baselines}
	\label{fig:vqbaselinesm}
\end{figure}


\section{Conclusions and Managerial Implications}

Using the Cox proportional hazards model along with appropriately chosen internal and external variables led to accurate retirement predictions. In the training sample, prediction error as measured by MAPE was approximately 25\% while the predictions in the smaller two-year holdout window were approximately 5\%. Although including only two validation points, these results indicate that the method has favorable potential.

The key internal variables that improved model predictions include division, years of service at hire (YCSH), and age at hire (AGEH).  In addition, S\&P 500 real earnings were also significantly associated with risk of retirement although the magnitude at approximately 1.3\% was (greater or less) than for small changes' other effects. The retirement hazard increases significantly when an individual accumulates 85 points of retirement credit and is eligible for full benefits but then reverts to a lower hazard after age 65. Furthermore, the early retirement incentive plan that the organization implemented in 2008 had a major impact with a large increase in the hazard of retirement, particularly for those eligible for full benefits with points exceeding 85. Considering individual and external information in out-of-sample tests provides a significant improvement over more traditional forecasting methods  \citet{feldman1994}. Such information also provides useful predictions on subgroups that are not possible with standard forecasting models.

Quitting behavior differed significantly by division, occupation, and age at start of service, reflecting differences in both worker satisfaction across the organization and by job type. Age connects logically because for a given number of years of service, the employee who starts earlier is younger and, therefore, may see a more significant long-term opportunity in a new position elsewhere. The 2008 early-retirement incentive program was also correlated with a significant amount of quitting. It is unclear if the higher hazard of quitting resulted directly from disruptions, management, accelerated retirements, or the drastic economic changes during that time.  In terms of predictive power, the quitting model performs well with MAPE in the training sample estimated at 15.16\% and at 18.77\% in the holdout sample. Both estimates are far superior to those that traditional forecasting techniques generate. As with retirement, external information in the form of S\&P real corporate dividends were positively correlated with quitting, thus indicating that as large private-sector companies' profits increase, the hazard of quitting rises significantly.

Drawing on these findings, this work provides a number of valuable managerial implications. Foremost, this work shows that fairly accurate survival model-based retirement prediction is feasible in large organizations when defined benefit plans are in place. These models are accurate enough to provide useful predictive visibility for human resources management professionals. Such models are of particular value when a large portion of the workforce has specialized skills requireing an extensive search process to replace or if the organization requires long lead times in the hiring process because of security or other concerns. The model also provides guidance as to expected retirement and quitting behavior across the organization's subgroups.  Significant deviations from these expectations, particularly in terms of abnormal quitting behavior, may indicate the need for management intervention. Finally, such models can alert managers to key external factors that may indicate increased retirement or quitting behavior. Using lagged variables may allow management to craft incentive policies in response to changes in the external economy.

Although this study brings to light much, it is important to consider a number of limitations because of the data and models used. The data were sampled based on a specific time window leading to a truncated sample, which limits the modeling approaches. Thus, it would be beneficial to have a longer prediction window to validate the model. Variables such as salary and more carefully documented division data, which would be available within the sponsoring organization, could strengthen the results presented here. In the same vein, more complete internal organizational details on early retirement incentive program as well as the defined benefit plan's details could improve the inferences' quality in this study. Finally, the accuracy of results for this study's retirement portion are based on the specific nature of the defined benefit pension dominant during the study period. The accuracy of predictions observed in this study may not generalize to organizations offering defined contribution retirement plans although a similar methodology may still prove useful in generating predictions. Although tested, the proportional hazards assumption may be violated for some effects in terms of the model. In addition, the baseline estimator's variability can cause overestimation of hazard. Thus, more sophisticated smoothed baseline models may be beneficial in this case.

Using predictive models, this work contributes significantly to both practical and academic literature regarding factors affecting retirement and management of staffing.  Future work should continue developing and improving the modeling techniques and result in better explanatory models for academic research as well as better functioning systems for industry. Furthermore, it is believed that the current methods can be extended to include qualitative and survey-based longitudinal feedback from employees through the time-varying covariate methodology. The addition of such attitudinal data may provide significant value to both academic and industry practitioners.